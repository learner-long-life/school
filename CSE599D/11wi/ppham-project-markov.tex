\section{Markov Chain Mixing}

In Step 4 of the verification scheme, we apply a Markov chain
$\hat{B}$ and then
project onto its +1 eigenstates. This depends on valid money states
$\ket{\$_p}$
having eigenvalues very close to (but not exactly equal to) +1,
which is suggested by the previous
section. But are $\ket{\$_p}$ eigenvalues much closer than
the eigenvalues corresponding to invalid money?
Our only
hope is
%that the eigenvalues for $\ket{\$_p}$ being exponentially
%close to one and
that the ``invalid'' eigenvalues
are at least
polynomially farther away.

Unfortunately, we don't understand enough about knots to make that
claim for this particular Markov chain. This is the biggest open
question and avenue for attack in our knot-based scheme.
In particular, we don't know
the eigenvalue gap, if any, between the lowest eigenvalue of
a $\ket{\$_p}$ (call it $(1-a), a \in [0,1)$) and the highest eigenvalue of any other
eigenstates (call it $(1-b), b \in [0,1)$). We are guaranteed to be exponentially close
to some eigenstate of $\hat{B}$ after calculating and measuring the
Alexander polynomial in Step 2 above.

In our wildest dreams, what would we like to be true about the eigenvalues
of $\hat{B}$?
First off, we'd like $b > a$. Next, we'd like $a$ to be
exponentially small, so that a $\ket{\$_p}$ doesn't degrade under
$r$ repetitions of Markov chain verification and still projects
to a +1 eigenstate with high probability.
Finally, we'd like $b$ to be at least polynomially away from 1, so
that under $r$ repetitions of Markov chain verification, it
projects to a +1 eigenstate with low probability.

%\begin{eqnarray*}
%a & = & \frac{1}{\exp(\Omega(D))} \\
%b & = & \frac{1}{\Omega(D)}
%\end{eqnarray*}

How high is high and how low is low?
We would like to show that the difference in probabilities increases
exponentially close to 1 with $r$, where in the first inequality below
we use the union bound.

\begin{multline*}
(1-a)^r - (1-b)^r \ge (1 - ra) - (1 - rb) \\
= r(b-a) =
r \left(\tfrac{1}{\exp(\Omega(D))} - \tfrac{1}{\Omega(D)}\right)
\end{multline*}

Therefore, if $(b-a)$ also increases exponentially closer to 1, we can
get away with $r = \textrm{poly}(D)$ repetitions, so that our
Markov chain verification procedure is tractable.