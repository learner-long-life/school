\documentclass[12pt]{article}
\usepackage{scribe}
\usepackage[all]{xy}
\Scribe{Paul Pham}
\Lecturer{Aram Harrow}
\LectureNumber{7}
\LectureDate{31 January 2011}
\LectureTitle{Shor's Factoring Algorithm and QFT}
\begin{document}
\MakeScribeTop

\newcommand{\ket}[1]{|#1 \rangle}
\newcommand{\bra}[1]{\langle #1 |}

In this lecture, we'll show how to factor integers by the reductions below,
the last two of which are due to Peter Shor and together make up a quantum
algorithm for factoring, although most of it is still classical.

\begin{enumerate}
\item Reduce factoring to order finding (Miller '76).
\item Reduce order finding to period finding.
\item Reduce period finding to QFT and continued fractions.
\end{enumerate}

%------------------------------------------------------------------------------
\section{Preliminaries}

The problem of factoring (factorizing) a number $N$ is to find its prime factors
$p_i$ (possibly up to a power $\alpha_i$):

\begin{displaymath}
N = \prod_{i=1}^k p_i{\alpha_i}
\end{displaymath}

\subsection{GCD and Euclid}

A useful tool in factoring is finding the greatest common denominator
(denoted $gcd$), which has the following useful properties:

\begin{displaymath}
gcd(x,y) = gcd(x, y-x) = gcd(x, y - \lfloor \frac{y}{x} \rfloor)
\end{displaymath}

Euclid's algorithm does this efficiently in time polynomial in the number
of bits of the numbers $x$ and $y$.

\subsection{ModExp}

Another useful idea is modular exponentiation, $a^x \mod N$ for
$0 \le a,x < N$ which we also know how to do efficiently via successive
squaring based on the binary expansion of the exponent $x$:

\begin{displaymath}
a \rightarrow a^{2^{k_1}} \mod N \rightarrow a^{2^{k_2}} \mod N \rightarrow \ldots
\rightarrow a^{2^{k_l}} \mod N
\end{displaymath}

where $x$ has $l \le \log N$ bits:

\begin{displaymath}
x = 2^{k_1} + 2^{k_2} + \ldots + 2^{k_l}
\end{displaymath}

\subsection{Pre-Reduction: Finding a Square Modulo $N$}

As a preliminary reduction, suppose that we can find $x$ such that

\begin{displaymath}
x^2 \equiv 1 \mod N \rightarrow (x + 1)(x - 1) \equiv 0 \mod N
\end{displaymath}

excepting the trivial solutions $x = 1$ and $x = N-1$.
Therefore, $N$ must have a common factor with either $x-1$ or $x+1$, therefore
our desired non-trivial factor is either $gcd(x+1,N)$ or $gcd(x-1,N)$.
How do we find this $x$?

%------------------------------------------------------------------------------
\section{Order Finding}

First, we show how factoring a number $N$
reduces to order-finding after considering the
the cases below where we can trivially find factors before we bring out
any heavy machinery.

\begin{description}
\item[Even $N$]:
Then the factors of $N$ are 2 and $N/2$.
\item[$N$ is a power of some prime factor]

That is, $N = p^\alpha$.
We won't show this here, but there is a classical algorithm for
factoring in this case that runs in time $O(\log_2^3 N)$.
You can find this in Dave Bacon's lecture \#11
notes for the Winter 2006 version of this class.

\end{description}

Okay, so now on to more tricky cases of $N$.

Recall $ord_N(a) = \min_{r \ge 1} a^r \equiv 1 \mod N$.
This is undefined if $gcd(a,N) \ne 1$.

Now we make two assumptions, show how they allow us to factor, and
justify them later.

\begin{enumerate}
\item
Suppose we can find $r = ord_N(a)$ through some black box,
and therefore $a^r - 1 \equiv 0 \mod N$.
\item
Further suppose that $r$ is even.
\end{enumerate}

If $r$ is even, then $r/2$ is an integer, and we have the following:

\begin{displaymath}
a^r \equiv 1 \mod N \rightarrow
(a^{r/2}+1)(a^{r/2}-1) \equiv 0 \mod N
\end{displaymath}

That is, $N$ is a multiple of either $a^{r/2}-1$ only (Case 1), or
$a^{r/2}+1$ only (Case 2),
or both (Case 3). Let's consider them separately:

\begin{description}
\item[Case 1] We know $a^{r/2}-1 \not\equiv 0 \mod N$ because $r$ is the minimum
power such that $a^r \equiv 1 \mod N$, so this case can't actually happen.
\item[Case 2] if $a^{r/2}+1 \equiv 0 \mod N$, then this is a case which is bad
for us, in that I don't know if a solution exists. We'll later show that
the probability of this case is small. On to the next case!
\item[Case 3] if $a^{r/2}+1 \not\equiv 0 \mod N$, then $gcd(a^{r/2} \pm 1, N)$ are both
non-trivial factors of $N$, and we are done.
\end{description}

\subsection{Reduction of Factoring to Order Finding}

Now we can formulate our reduction:

\begin{enumerate}
\item Choose $a \in \{2, \ldots, N-1\}$ uniformly at random.
\item if $gcd(a,N) \ne 1$, then we are done.
\item Find $r = ord_N(a)$ using our black box.
\item If $r$ is odd, try again with another $a$.
\item If $r$ is even, calculate both $gcd(a^{r/2} \pm 1, N)$ and we're done.
\end{enumerate}

We will now justify one of the gaping assumptions above, namely that our
reduction completes in bounded time with some high probability of success.
That is, we want the probability of the cases where our reduction doesn't
work to be exponentially small in $k$, the number of prime factors of $N$.

\begin{displaymath}
\Pr\left[ r \text{ is odd or } a^{r/2} + 1 \equiv 0 \mod N \right] \le 1/2^{k-1}
\end{displaymath}

\subsection{Success Probability of Order-Finding Reduction}

First, consider the positive integers less than $N$ and coprime to
it:

\begin{displaymath}
\mathbb{Z}^*_N = \left\{ 1 \le x \le N : gcd(x,N) = 1 \right\}
\end{displaymath}

Note that the coprime condition stipulates
$\exists_{a,b} ax + bN \equiv 1 \mod N$, so that we can define
$x^{-1} \equiv a \mod N$.
Since the set above is closed under integer multiplication modulo $N$ and
every element has an inverse, this is a group.

By the Chinese Remainder Theorem, we can decompose elements of
$\mathbb{Z}^*_n$ as unique products of elements from the cyclic groups of
its prime factor powers.

\begin{displaymath}
\mathbb{Z}^*_N = \mathbb{Z}^*_{p_1^{\alpha_1}} \times \ldots
\mathbb{Z}^*_{p_k^{\alpha_k}}
\end{displaymath}

That is, $a \in \mathbb{Z}^*_N$ is uniquely determined by
$a \mod p_1^{\alpha_1}, \ldots , a \mod p_k^{\alpha_k}$.

We have the size of this group as

\begin{displaymath}
\mid \mathbb{Z}^*_N \mid = (\prod_{i=1}^k (1 - \frac{r}{p_i})) N
\end{displaymath}

The order of $a$
with respect to $N$ is the least common multiple of $a$'s order with
respect to each $p_i^{\alpha_i}$

\begin{displaymath}
ord_N(a) = lcm(ord_{p_1^{\alpha_1}}(a), \ldots , ord_{p_k^{\alpha_k}}(a))
\end{displaymath}
\begin{displaymath}
r = ord_N(a) \qquad r = lcm(r_1, \ldots r_k)
\end{displaymath}

\begin{displaymath}
r_i = ord_{p_i^{\alpha_i}}(a) \qquad d_j \equiv \max_{d} 2^d \mid r_j
\end{displaymath}

\begin{displaymath}
d_1 = \ldots = d_k = 0 \qquad \Longleftrightarrow \qquad
\text{all } r_i \text{ odd} \qquad \Longleftrightarrow \qquad r \text{ odd}
\end{displaymath}

Claim:

\begin{displaymath}
a^{r/2} + 1 \equiv 0 \mod N \Longleftrightarrow d_1 = \ldots = d_k > 0
\end{displaymath}

\begin{displaymath}
a^{r/2} + 1 \equiv 0 \mod N \Longleftrightarrow
a^{r/2} + 1 \equiv 0 \mod p_i^{\alpha_i} \forall{i \in [k]}
\end{displaymath}

Define $d$ to be the following:

\begin{displaymath}
d \equiv \max \left\{ d_i : 2^{d_i} \mid r \right\}
\end{displaymath}

\begin{displaymath}
d_i < d \rightarrow r_i \mid \frac{r}{2}
\end{displaymath}

Choose $a$ uniformly from the following group:

\begin{displaymath}
a \in \mathbb{Z}^*_{p_i^{\alpha_i}} \cong
\mathbb{Z}^+_{p_i^{a_i}}(1 - \frac{1}{p_i})
\end{displaymath}

Define $b_i$ to be the following:

\begin{displaymath}
b_i = \max \{b_j : 2^{b_j} \mid p_i^{\alpha_i}(1 - \frac{1}{p_i})
\end{displaymath}

\begin{eqnarray*}
\Pr[d_i = 0] & = & \frac{1}{2}\\
\Pr[d_i = 1] & = & \frac{1}{4}\\
\ldots \\
\Pr[d_i = b_i - 1] & = & 2^{-b_i}\\
\Pr[d_i = b_i] & = & 2^{-b_i}
\end{eqnarray*}

This is the low probability that we want.
Now on to period finding.

%------------------------------------------------------------------------------
\section{Period Finding and QFT}

Let

\begin{displaymath}
f(x) = a^x \mod N
\end{displaymath}

\begin{multline}
f(x) = f(x+r)\\
f(y) = f(x) \Longleftrightarrow r \mid x-y \\
0 \le x < 2^n \qquad 2^n \gg N^2
\end{multline}

where we can think of $n \sim O(\log N)$.

Now here is the reduction to the Quantum Fourier Transform, which as you recall
is defined by:

\begin{displaymath}
F = \frac{1}{\sqrt{2^n}} \sum_{x,y=0}^{2^n - 1}
e^{2\pi i xy / 2^n} \ket{y} \bra{x}
\end{displaymath}

\begin{enumerate}
\item Start with $\frac{1}{\sqrt{2^n}} \sum_{x=0}^{2^n - 1} \ket{x}$ in the
first register.
\item Apply $f$, storing the result in a second register,
obtaining $\frac{1}{\sqrt{2^n}}\sum \ket{x} \ket{f(x)}$.
\item Measure the second register to obtain the following state for some $0 \le x < r$

\begin{displaymath}
\frac{\ket{x} + \ket{x+r} + \ket{x + 2r} + \ldots + \ket{x + (l-1)r}}{\sqrt{l}} =
\frac{1}{\sqrt{l}} \sum_{j=0}^{l-1} \ket{x + jr}
\end{displaymath}

\item Apply $F$ to get
\begin{displaymath}
\frac{1}{\sqrt{2^n l}} \sum_{j=0}^{l-1} \sum_{y=0}^{2^n - 1}
\exp(2\pi i y (x + jr) / 2^n \ket{y}
\end{displaymath}

\item Measure $y$ with the following probability:

\begin{displaymath}
\Pr[y] = \frac{1}{2^nl} \mid \sum_{j=0}^{l-1} \exp(2 \pi i y j r / 2^n) \mid ^2
\cdot \mid \exp(2 \pi i y x) \mid ^2
\end{displaymath}

If $y \approx \frac{2^n}{r}k$, for $0 \le k < r$, $k \in \mathbb{Z}$, then
this probability is: $\frac{2 \pi i y j r}{2^n} \approx 2 \pi i j k$.

\end{enumerate}

Claim:

$\Pr[y]$ peaks around $y \in \left\{ 0, \frac{2^n}{r}, \frac{2^n}{r}2,
\frac{2^n}{r}3, \ldots \right\} \}$

We're not going to prove this, but it is similar to the homework problem for
phase estimation. $f$ repeats with a period $r$, like a unitary which
repeats phase with a period $r$.

Define:

\begin{displaymath}
X_r \equiv \sum_{j=0}^{r-1} \ket{j+1 \mod r} \bra{j}
\end{displaymath}

The eigenvalues of this operator are $e^{2\pi i k / r}$ for
$k = 0, 1, \ldots, r-1$, for the corresponding eigenvectors:

\begin{displaymath}
\ket{\psi_k} = \frac{1}{\sqrt{r}} \sum_{j=0}^{r-1} e^{-\frac{2\pi i j k}{r}\ket{j}}
\end{displaymath}

After running phase estimation, we get some phase approximately equal to
$\frac{k}{r}$ for some random $k$. Now run continued fractions expansion to
get the best rational approximation to $r$.
See Nielsen \& Chuang.

We can write the result of a QFT using the
following \emph{product representation}, a tensor product of single qubit
states:

\begin{displaymath}
F\ket{x_0, \ldots x_{n-1}} =
\frac{\ket{0} + e^{2\pi i \cdot 0.x_{n-1}\cdots x_0} \ket{1}}{\sqrt{2}} \otimes
\frac{\ket{0} + e^{2\pi i \cdot 0.x_{n-2}\cdots x_0} \ket{1}}{\sqrt{2}} \otimes
\ldots
\end{displaymath}

Therefore, we can enact the QFT by doing controlled rotations on single-qubits,
which look like:

\begin{displaymath}
R_k = 
 \left(
  \begin{array}{cc}
    1 & 0 \\
    0 & e^{2\pi i / 2^k} \\
  \end{array} \right)
\end{displaymath}

\end{document}
