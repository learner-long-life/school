\documentclass[twocolumn,10pt]{article}

\usepackage[dvips]{graphicx}
%\usepackage{times}
%\usepackage{fullpage}
\usepackage{eprint}
\usepackage{rotating}
%\usepackage{eepic}
\usepackage{amsfonts}
\usepackage{algorithmic}
\usepackage{amsthm}
\usepackage{amsmath}

\theoremstyle{plain}
\newtheorem{theorem}{Theorem}

\title{Quantum Money From Knots\\
\large{CSE 599D Winter 2011: Final Project}
}
\date{18 March 2011}
\author{Paul Pham}

%\input{Qcircuit}

\begin{document}

\newcommand{\ket}[1]{|#1 \rangle}
\newcommand{\bra}[1]{\langle #1 |}
\newcommand{\braket}[2]{\langle #1|#2 \rangle}
\newcommand{\normtwo}{\frac{1}{\sqrt{2}}}
\newcommand{\norm}[1]{\parallel #1 \parallel}
\newcommand{\Dhalf}{\tfrac{D}{2}}
\newcommand{\threeDhalf}{\tfrac{3D}{2}}
\newcommand{\cutoffInterval}{\overline{\left[\Dhalf, \threeDhalf\right]}}
\newcommand{\cutoffG}{\mathcal{G}_p(\cutoffInterval)}

\maketitle

\section{Abstract}

In this project, I critique the public-key quantum money scheme of
\cite{Farhi2010} based on the hardness of finding equivalent
mathematical knots and creating a weighted superposition of
corresponding grid diagrams that correspond to the same Alexander
polynomial.
I sketch a loose upper bound for the damage to a valid
money state from the verification procedure and characterize the
desired behavior of Markov chain mixing in order to distinguish
valid and invalid money states with a polynomial number of trials.
In conclusion, I propose extensions to this work to create
a more concrete scheme and to defend against future attacks.

\section{Introduction}

General quantum states cannot be cloned, but this apparent
algorithmic disadvantage can be turned into a cryptographic advantage.
Money is a common implementation of a
real-life, physical one-way function:
we want valid bills and coins to be easily creatable (via a
\emph{minting algorithm} possibly with some classical secret) by a central
bank but easily checkable (by a public \emph{verification algorithm})
by anyone with access to a quantum computer.

This project provides a critical summary of a recent proposed 
 quantum money scheme based on the properties
mathematical knots \cite{Farhi2010}.
The interested reader is referred to that paper for
a good summary of prior work.
While a promising approach, this scheme's Markov-chain-based
verification algorithm is incomplete
and may not be able to distinguish valid and invalid quantum money states.

First, we briefly review knots and how they are used
in the minting algorithm to create valid money states.
Second, we move on to the main part of this paper, the dissection of the
verification algorithm, including a calculation bounding how much
damage is done to a valid money state and a discussion about our
desired mixing properties for the Markov chain part.
Then we expand upon existing attacks for this scheme.
Finally, we conclude with future extensions to this exciting work to
make the scheme more concrete.

%\section{Related Work}

%The unforgeability of quantum money was studied as early as Wiesner
%\cite{}. Although his scheme provides information-theoretic security
%in the sense of relying directly on the laws of physics, it has the
%severe disadvantage of involving the mint in every transaction.
%Ideally, we would like our quantum money to be publicly verifiable, that is,
%without resorting to the trusted authority for every interaction.
%Aaronson proved that public-key quantum money was possible relative to an oracle




\section{Quantum Money}

Consider notes which consist of a quantum state $\ket{\$_p}$ in
a fixed basis $\mathcal{B}$
and a classical serial number $p$,
together with a global classical function $A$ which computes
$p$ from the basis states. $\ket{\$_p}$ is a (possibly weighted) superposition
of basis states $\ket{G}$ which are equivalent in the sense that they all map
to the same value.

\begin{displaymath}
\ket{\$_p} = \frac{1}{\sqrt{N}} \sum_{G:A(G)=p} q_G \ket{G}
\end{displaymath}

With respect to some security parameter $\overline{D}$, 
we would like a different serial number $p$ to be produced each time
with probability exponentially close to 1 to prevent forgery through
repetition
of the minting algorithm. For each $p$, or even from each $\ket{G}$ where
$A(G)=p$, it should be difficult to find all equivalent basis states and
therefore
to forge their superposition. We will see later that it is useful to
shape the weights $q_G$ of the superposition, rather than have uniform
distribution. So how do we do this? Two words: knots, maybe.

\section{Knots}

Knots are like a loop of string which can be arbitrarily tangled
with
itself in three dimensions. We
can represent them in two-dimensions with $d \times d$
\emph{grid diagrams}, where strands
pass vertically and horizontally between $d$ \textsf{X} and
$d$ \textsf{O} markers,
one of each kind of marker in each column and row.
Equivalently, we can encode a grid diagram purely through
a pair of disjoint permutations $\Pi_{\textsf{X}}$
and $\Pi_{\textsf{O}}$ on the integers $\{1, \ldots, d\}$.

An Alexander polynomial can be computed for each knot based on its
crossings and is invariant under the Reidemeister moves. Therefore,
all grid diagrams of the same knot have the same Alexander polynomial
and can be transformed into one another.
However, it is conjectured to be hard (even for a quantum computer, on
average) to be able to generate all such equivalent
grid diagrams, or even more simply to determine if two grid diagrams are
equivalent. However, we can easily create a quantum
superposition of all grid encodings, compute their Alexander polynomials
$p$ into a second register, then measure $p$ to be left with the
superposition of all corresponding equivalent grid diagrams.
This one-wayness, due to quantum measurement, combined with the classical
one-wayness of digitally signing $p$, result in the one-wayness of
the minting algorithm.

Based on the leading notation above, you have probably guessed that the
basis $\mathcal{B}$ consists of all grid diagrams of size $d$ which ranges
from $2$ to $2\overline{D}$. The size of a grid diagram is
denoted as $d(G)$.

\section{Knots}

Knots are like a loop of string which can be arbitrarily tangled
with
itself in three dimensions. We
can represent them (non-uniquely) in two-dimensions with $d \times d$
\emph{grid diagrams}.
%, where strands
%pass vertically and horizontally between $d$ \textsf{X} and
%$d$ \textsf{O} markers,
%one of each kind of marker in each column and row.
%Equivalently, we can encode a grid diagram purely through
%a pair of disjoint permutations $\Pi_{\textsf{X}}$
%and $\Pi_{\textsf{O}}$ on the integers $\{1, \ldots, d\}$.
An Alexander polynomial can be computed for each knot based on its
crossings and is invariant under the Reidemeister moves on all
corresponding gird diagrams.

Based on the leading notation above, you have probably guessed that we
choose our basis $\{G\}$ to consist of all grid diagrams of size
within the arbitrarily chosen range
$[2,2D]$.
Furthermore, we choose $A$ to compute
the Alexander polynomial $p$ of a graph $G$.
We denote the integer size of a grid diagram $G$ by $d(G)$.
Now it is useful to restrict our definition of equivalent basis states
to grid diagrams representing the same underlying knot,
with sizes within a certain interval $[a,b]$ for $a,b \in \mathbb{Z}^+$,
with omission of an interval meaning the
maximum range: $\mathcal{G}_p = \mathcal{G}_p([2,2D])$.

\begin{displaymath}
\mathcal{G}_p([a,b]) = \{G: (A(G) = p) \land (d(G) \in[a,b]) \}
\end{displaymath}

It is conjectured to be hard on average (even for a quantum computer)
to be able to generate all $G \in \mathcal{G}_p([a,b])$ given
uniformly random $p$, $a$, and $b$. We will skip the minting algorithm,
but its one-wayness is based on the random outcome of measuring $p$
out of a superposition of $\{\mathcal{G}_p\}$, together with a classical
private key for signing valid serial numbers.

%This one-wayness, due to quantum measurement, combined with the classical
%one-wayness of digitally signing $p$, result in the one-wayness of
%the minting algorithm. We can then digitally sign $p$ with the bank's
%private key and publish that along with $(\ket{\$_p},p)$.
%However, we can easily create
%$\ket{\$_p}$ for a random $p$ via our minting algorithm:

%\begin{enumerate}
%\item Create a quantum
%superposition of all grid diagrams $\ket{G}$ in a first register.
%\item Compute their Alexander polynomials
%$p$ into a second register.
%\item Measure $p$ to be left with the
%superposition of all $G \in \mathcal{G}_p$.
%corresponding equivalent grid diagrams.
%\end{enumerate}

\section{Superposition Weights}
\label{sec:weights}

So how do we verify a valid money state $\ket{\$_p}$ made up of possibly
exponentially many, hard-to-find
equivalent grid diagrams? Note that
the local moves to transform one grid diagram into another can be
compactly represented as a Markov chain. Simply allow a given state to
mix according to this Markov chain for long enough. If we started
out with a valid $\ket{\$_p}$, then mixing won't change it because
all $G\in\mathcal{G}_p$ will mix with each other in isolation from the
grid diagrams of another polynomial $p'$. If its
eigenvalue is close to +1, it should project to the
stationary distributions (the +1 eigenstates) with high probability.

However, there is one big wrinkle in that we have to limit ourselves
to a finite size interval,
so any moves which go outside that will not happen in our Markov chain.
There may be small subsets of $\mathcal{G}_p$, which we'll call
\emph{shards},
close to that size limit that will
not mix as a result, which could be
easy to forge but still pass verification.
To get around this, we can define our superposition
weights (and our corresponding Markov chain moves) to heavily favor
grid diagrams close to some mean and then
cut off the tails. Therefore,
we define our weights $q_G=q(d(G))$ in Section \ref{sec:model}
to be
a Gaussian distribution on the grid diagram's size 
centered at $D$ with standard deviation
$\sqrt{D}$.

\begin{displaymath}
q(d) = 
\begin{cases} 
\lceil \tfrac{1}{N} \exp \tfrac{-(d-D)^2}{2D} \rceil & \text{if } 2 \le d \le 2D\\ 
0 & \text{otherwise }
\end{cases} 
\end{displaymath}

The distribution is integer-valued, and the Markov chain walks over
configuration pairs $(G,i)$ of diagram grid $G$ and a label
$i \in [q(d(G))]$. Moves that takes us from $G$ to $G'$ where
$i \ge q(d(G'))$ are not allowed.
Now we show that cutting of the tails doesn't damage our state too much.

\section{Verification}

Now we're ready to tackle 
the main steps of verifying whether a pair $(\ket{\phi},p)$ is valid
quantum knot money as follows. Keep in mind that
rejection at any step results in
the whole procedure rejecting.
%, and possibly a visit from Treasury agents.

\begin{enumerate}
\item
Verify that $\ket{\phi}$ is a superposition of valid encoded grid diagrams.
\item
Measure the Alexander polynomial on the state $\ket{\phi}$ and verify
that it is equal to $p$.
At this point, $\ket{\phi}$ is some superposition of
$\mathcal{G}_p$, but may not have the correct weights.
\item
Measure the projector onto grid diagrams with size in the range
$\left[ \Dhalf, \threeDhalf \right]$.
Verify that the outcome is $+1$.
\item
Apply $r$ trials of
Markov chain verification. Verify that all trials have outcome $+1$.
\end{enumerate}

We will now discuss the motivation and important implications of the
last two steps.

%Two important things are worth noting about the last two steps, both of
%which we
%will examine in greater detail: the estimated damage done to a valid money
%state by projecting it to a smaller subspace,
%and how well the mixing properties of the Markov chain allow us
%to distinguish valid from invalid money states.

\section{Measurement Damage}
\label{sec:damage}

In Step 3, we are cutting off the tails of the Gaussian distribution,
like mice in some gruesome evolution experiment,
to eliminate the shards mentioned in Section \ref{sec:weights}.
 These edge cases will not mix with the rest of
$\mathcal{G}_p$ and may have artificially small size, making them easy
to forge and still pass Step 4.
How do we know this projection won't significantly damage a
valid money state?

First let's define the cut-off interval as the complement of the
``cut-in'' interval:

\begin{displaymath}
\cutoffInterval = \left[2, \Dhalf \right)
\cup \left(\threeDhalf,2D\right]
\end{displaymath}

So the grid diagrams in the cut-off interval are:

\begin{multline*}
\cutoffG
= \\ \{
G: (A(G)=p) \land (d(G) \in
[2, \Dhalf)
\cup (\threeDhalf,2D])
\}
\end{multline*}

Then lets take the norm of the difference between the original valid money state
$\ket{\$_p}$ and the same state after its tails have been cut off,
$\ket{\tilde{\$_p}}$, which ends up just being the sum of the coefficients
squared
of $G \in \mathcal{G}_p$.

\begin{eqnarray*}
|| \ket{\$_p} - \ket{\tilde{\$_p}} || & = &
\sum_{G \in \cutoffG} (\sqrt{q(d(G))})^2 \\
& = & \sum_{d \in \cutoffInterval} \sum_{G \in \mathcal{G}([d])} q(d(G)) \\
& = & \sum_{d \in \cutoffInterval} |\mathcal{G}_p([d])| q(d) \\
& \le & |\mathcal{G}_p|_{\textrm{max}}
\sum_{d \in \cutoffInterval} q(d)
\end{eqnarray*}

In the last inequality, we upper bound using the maximum size of
$\mathcal{G}_p$ for any $d$ in the cut-off interval.

\begin{displaymath}
|\mathcal{G}_p|_{\textrm{max}} =
(\textrm{argmax}_{d \in \cutoffInterval}|\mathcal{G}_p([d])|)
\end{displaymath}

Recall that $q(d)$ is designed to be a Gaussian distribution with
standard deivation $\sqrt{D}$, which we can recenter to zero mean.
Then we
can calculate the area under the distribution $q(d)$ from
$\frac{D}{2}$ to $\frac{3D}{2}$ using the error function:
\footnote{\url{http://en.wikipedia.org/wiki/Normal\_distribution\#Standard\_deviation\_and\_confidence\_intervals}}

\begin{multline*}
F(\mu + n\sigma; \mu, \sigma^2) - F(\mu - n\sigma; \mu, \sigma^2) = 
\Phi(n) - \Phi(-n) \\ = \textrm{erf}(\tfrac{n}{\sqrt{2}}) =
\textrm{erf}(\tfrac{\sqrt{D}}{2\sqrt{2}})
\end{multline*}

Here, $n = \sqrt{D}$, and we can approximate the
error function by:

\begin{displaymath}
\textrm{erf}(\tfrac{\sqrt{D}}{\sqrt{2}}) =
\sqrt{1 - \exp(-\Omega(D^2))} = 1 - \exp(\Omega(-D^2))
\end{displaymath}

Assuming that $|\mathcal{G}_p|_{\textrm{max}}| = \textrm{poly}(D)$,
then we conclude that the damage to a valid money state is exponentially
small.

\begin{displaymath}
|| \ket{\$_p} - \ket{\tilde{\$_p}} || = 1 - \exp(\Omega(-D^2))
\end{displaymath}

%For some intuition about why our assumption
%might be true, consider the following naive upper bound on the 
%For a given size $d$, we have $O(d^2)$ moves which can increase the size
%of the grid to $d+1$, and each of those resulting $d+1$ size grids can
%have $O(d^2)$ moves that maintain the same size. Sum $O(d^4)$ for $d$ in
%some interval linear in $D$ to get a polynomial in
%$D$.

\section{Markov Chain Mixing to Distinguish Money States}

In Step 4 of the verification scheme, we apply a Markov chain
$\hat{B}$ and then
project onto its +1 eigenstates. This depends on valid money states
$\ket{\$_p}$
being very close to +1 eigenstates of $\hat{B}$, much closer than
the eigenstates corresponding to invalid money. 
Valid money states cannot be exactly +1 eigenstates, because those
include mixing from grid diagrams in $\mathcal{G}$ above,
with size in the tails that we cut off in Step 3. Therefore, our only
hope is that the eigenvalues for $\ket{\$_p}$ being exponentially
close to one and the eigenvalues for all other states being at least
polynomially farther away.

Unfortunately, we don't understanding enough about knots to make that
claim for this particular Markov chain. This is the biggest open
question and avenue for attack in our knot-based scheme.
In particular, we don't know
the eigenvalue gap, if any, between the lowest eigenvalue of
a $\ket{\$_p}$ (call it $(1-a), a \in [0,1)$) and the highest eigenvalue of any other
eigenstates (call it $(1-b), b \in [0,1)$). We are guaranteed to be exponentially close
to some eigenstate of $\hat{B}$ after calculating and measuring the
Alexander polynomial in Step 2 above.

However, as dreamers, we can imagine what desirable properties we would
like to prove for $\hat{B}$. First, we would like $b > a$, so that
there is a gap. First, we would like to show that $a$ is
small, so a $\ket{\$_p}$ doesn't degrade under
$r$ repetitions of Markov chain verification and still projects
to a +1 eigenstate with high probability.

\begin{displaymath}
a = \frac{1}{\exp(\Omega(\overline{D}))}
\end{displaymath}

Second, we would like to show that $b$ is polynomially away from 1, so
that under $r$ repetitions of Markov chain verification, it
projects to a +1 eigenstate with low probability.

\begin{displaymath}
b = \frac{1}{\Omega(\overline{D})}
\end{displaymath}

We would like to show that difference in probabilities increases
exponentially close to 1 with $r$:

\begin{multline*}
(1-a)^r - (1-b)^r \ge (1 - ra) - (1 - rb) \\
= r(b-a) =
\frac{1}{\exp(\Omega(\overline{D}))} - \frac{1}{\Omega(\overline{D})}
\end{multline*}

Therefore, if $(b-a)$ also increases exponentially closer to 1, we can
get away with $r = \textrm{poly}(\overline{D})$ repetitions, so that our
Markov chain verification procedure is tractable.

\section{Future Extensions}

To extend this scheme or prove it secure, we would need a better
understanding of knots and quantum algorithms for them. The
two obvious future extensions are to come up with a quantum algorithm
for knot equivalence or
to prove an eigenvalue gap for the Markov chain in the verification
scheme. Aside from those, here are a
few other interesting directions:

\begin{enumerate}
\item
For a given security parameter $D$, 
are there sufficiently many Alexander polynomials (serial numbers)
available to meet the global demand for quantum bills?
To do this, we would need to lower-bound $|\{\mathcal{G}_p\}|$,
not including the unknot which has empirically been shown
to
occupy the vast majority of grid diagrams.
\item
Quantum bills currently have no denomination associated with
them and so are of unit value.
Is it possible to associate a denomination with quantum money, or to have
it be dividable or combinable?
\item 
It turns out that a certain class of states can be efficiently copied, which
includes the eigenstates of the addition operator, $\ket{\psi_{n,k}}$,
used to enact arbitrary controlled phase
rotations in the quantum compiling algorithm of Kitaev, Shen, and Vyalyi
\cite{KSV02}.
%The state to copy (say $\ket{\psi_{n,1}}$, which is hard
%to produce), but is of the
%same form as the ``empty'' register to hold the new copy
%($\ket{\psi_{n,0}}$, which
%is easy to produce).
%As in
%\begin{displaymath}
%$\ket{\psi_{n,1}}\otimes\ket{\psi_{n,0}} \rightarrow
%\ket{\psi_{n,1}}\otimes\ket{\psi_{n,1}}$.
%\end{displaymath}
If it turns out that $\ket{\$_p}$ or some part of it is
within that class of states, then we can forge it with non-negligible
probability.
\end{enumerate}

Results which have emerged since the main paper \cite{Farhi2010}
include a new online attack for Wiesner's original scheme
which involves the bank returning bogus bills,
also addressed in \cite{Lutomirski2010} by 
\emph{collision-free} quantum money
protocols which prevent currency inflation.
Furthermore, if deciding knot equivalence in two dimensions is ever
solved, one might hope that lifting knots into
three-dimensional \emph{cube diagrams}
will evade solution for longer \cite{Baldridge2009}.

In conclusion, our hopes are raised by the promise of
this new public-key quantum money scheme, but we also poke at some of
its shortcomings. We've only scratched the surface of this fascinating
topic, leaving out many bad jokes, and
we look forward to future developments in this field.


\bibliography{ppham-project}
\bibliographystyle{tocplain}

\end{document}
