\section{Damage from Measuring Valid Money States}

In Step 3, we are cutting off the tails of the Gaussian distribution
to eliminate a certain class of equivalent knot diagrams with size close
to $2\overline{D}$ or $2$ which are easy to create due to the edge cases
in our Markov chain moves. Otherwise, these easily forgeable states would
pass Step 4. How do we know this projection won't significantly damage a valid
money state?

First let's define the set of all equivalent
grid diagrams $G$ with Alexander polynomial $p$ and with dimension outside
the cutoff regions:

\begin{displaymath}
\mathcal{G} = \{
G:A(G)=p \land d(G) \in
[2, \frac{\overline{D}}{2})
\cap (\frac{3\overline{D}}{2},2\overline{D}]
\}
\end{displaymath}

Then lets take the norm of the difference between the original valid money state
$\ket{\psi}$ and the same state after its tails have been cut off,
$\ket{\tilde{\psi}}$, which ends up just being the sum of the coefficients
squared
of grid diagrams in $\mathcal{G}$.

\begin{displaymath}
|| \ket{\psi} - \ket{\tilde{\psi}} || \le
\sum_{G \in \mathcal{G}} (\sqrt{q(d(G))})^2
\end{displaymath}

Recall that $q(d)$ is designed to be a Gaussian distribution with
standard deivation $\sqrt{\overline{D}}/2$, which we can recenter to zero mean.
Then we
can calculate the area under the distribution from
$\frac{\overline{D}}{2}$ to $\frac{3\overline{D}}{2}$ using the error function:
\cite{WikipediaNormal}

\begin{multline*}
F(\mu + n\sigma; \mu, \sigma^2) - F(\mu - n\sigma; \mu, \sigma^2) = 
\Phi(n) - \Phi(-n) \\ = \textrm{erf}(\frac{n}{\sqrt{2}}) =
\textrm{erf}(\frac{\sqrt{\overline{D}}}{2\sqrt{2}})
\end{multline*}

Here, $n = \sqrt{\overline{D}}/2$, and we can approximate the
error function by:

\begin{displaymath}
erf(\frac{\overline{D}}{2\sqrt{2}}) =
\sqrt{1 - \exp(-\Omega(\overline{D}^2))} = 1 - \exp(\Omega(\overline{D}^2))
\end{displaymath}
