\section{Future Extensions}

To extend this scheme or prove it secure, we would need a better
understanding of knots and quantum algorithms for them. The
two obvious future extensions are to come up with a quantum algorithm
for knot equivalence or
to prove an eigenvalue gap for the Markov chain in the verification
scheme. Aside from those, here are a
few other interesting directions:

\begin{enumerate}
\item
For a given security parameter $D$, 
are there sufficiently many Alexander polynomials (serial numbers)
available to meet the global demand for quantum bills?
To do this, we would need to lower-bound $|\{\mathcal{G}_p\}|$,
not including the unknot which has empirically been shown
to
occupy the vast majority of grid diagrams.
\item
Quantum bills currently have no denomination associated with
them and so are of unit value.
Is it possible to associate a denomination with quantum money, or to have
it be dividable or combinable?
\item 
It turns out that a certain class of states can be efficiently copied, which
includes the eigenstates of the addition operator, $\ket{\psi_{n,k}}$,
used to enact arbitrary controlled phase
rotations in the quantum compiling algorithm of Kitaev, Shen, and Vyalyi
\cite{KSV02}.
%The state to copy (say $\ket{\psi_{n,1}}$, which is hard
%to produce), but is of the
%same form as the ``empty'' register to hold the new copy
%($\ket{\psi_{n,0}}$, which
%is easy to produce).
%As in
%\begin{displaymath}
%$\ket{\psi_{n,1}}\otimes\ket{\psi_{n,0}} \rightarrow
%\ket{\psi_{n,1}}\otimes\ket{\psi_{n,1}}$.
%\end{displaymath}
If it turns out that $\ket{\$_p}$ or some part of it is
within that class of states, then we can forge it with non-negligible
probability.
\end{enumerate}

Results which have emerged since the main paper \cite{Farhi2010}
include a new online attack for Wiesner's original scheme
which involves the bank returning bogus bills,
also addressed in \cite{Lutomirski2010} by 
\emph{collision-free} quantum money
protocols which prevent currency inflation.
Furthermore, if deciding knot equivalence in two dimensions is ever
solved, one might hope that lifting knots into
three-dimensional \emph{cube diagrams}
will evade solution for longer \cite{Baldridge2009}.

In conclusion, our hopes are raised by the promise of
this new public-key quantum money scheme, but we also poke at some of
its shortcomings. We've only scratched the surface of this fascinating
topic, leaving out many bad jokes, and
we look forward to future developments in this field.
