\section{Future Extensions}

To extend this scheme or prove it secure, we would need a better
understanding of knots and quantum algorithms for them. The
two obvious future extensions are to come up with a quantum algorithm
for knot equivalence to attach the security of this scheme directly or
to prove an eigenvalue gap for the Markov chain in the verification
scheme. Aside from those, here are a
few other interesting directions:

\begin{enumerate}
\item Current schemes do not address the demand for currency.
For a given security parameter $\overline{D}$, 
are there sufficiently many Alexander polynomials (serial numbers)
available to supply the world with enough quantum bills?
To do this, we would need to lower-bound the number of
different knots that can be embedded in grid diagrams of up to size
$2\overline{D}$, not including the unknot which has empirically been shown
to
occupy the vast majority of grid diagrams.
\item
Quantum bills and coins currently have no denomination associated with
them and so are of unit value.
Is it possible to associate a denomination with quantum money, or to have
it be dividable or combinable?
\item 
It turns out that a certain class of states can be efficiently copied, which
includes the eigenstates of the addition operator, $\ket{\psi_{n,k}}$,
used to enact arbitrary controlled phase
rotations in the quantum compiling algorithm of Kitaev, Shen, and Vyalyi
\cite{KSV02}. The state to copy (say $\ket{\psi_{n,1}}$, which is hard
to produce), but is of the
same form as the "empty" register to hold the new copy
($\ket{\psi_{n,0}}$, which
is easy to produce). As in
\begin{displaymath}
\ket{\psi_{n,1}}\otimes\ket{\psi_{n,0}} \rightarrow
\ket{\psi_{n,1}}\otimes\ket{\psi_{n,1}}
\end{displaymath}
If it turns out that weighted superpositions of equivalent
grid diagrams are within that class of states, then the valid money state
$\ket{\$_p}$ could be copied into an entangled pair
$\ket{\$_p} \otimes \ket{\$_p}$, either have of which would pass
verification.
\end{enumerate}

Interesting results which have emerged since the main paper \cite{Farhi2010}
include a new online attack for Wiesner's original scheme
\cite{Lutomirski2010} which involves the bank returning bogus bills.
Incidentally, the related work \cite{Lutomirski 2010} addresses this same
concern of a mint artificially inflating currency by releasing additional bills,
and then sketches a solution using a \emph{collision-free} quantum money
protocol.

However, if one is not satisfied with the hardness of finding a connecting
sequence of Reidemeister moves between any two grid diagrams, we can increase
the hardness even further by embedding knots into three dimensions, using
so-called \emph{cube diagrams}\cite{Baldridge2009}.
