\section{Introduction}

General quantum states cannot be cloned. Is it possible to
turn this apparent disadvantage into a cryptographic advantage,
implementing money that is hard to counterfeit?
We want valid money tokens to be easily \emph{minted}
(possibly with some classical secret) by a central
bank but easily \emph{verified}
by anyone (with access to a quantum computer).

This project provides a critical summary of a recently proposed 
 quantum money scheme based on the properties
of knots and grid diagrams \cite{Farhi2010}.
The interested reader is referred to that paper for
a good summary of prior work.
While a promising approach, this scheme's Markov-chain-based
verification algorithm is incomplete.
Therefore our criticism will focus on the ability to
distinguish valid and invalid
quantum money states.

Our report is organized as follows.
First, we briefly review knots and how they are used
in the minting algorithm to create valid money states.
Second and most importantly, we analyze the relative ability of
valid and invalid money states to survive the
verification algorithm, specifically measurement damage and
the
desired mixing properties of the given Markov chain. This section
requires some ``reasonable'' assumptions, which in turn point to
possible future extensions, where we conclude.

%\section{Related Work}

%The unforgeability of quantum money was studied as early as Wiesner
%\cite{}. Although his scheme provides information-theoretic security
%in the sense of relying directly on the laws of physics, it has the
%severe disadvantage of involving the mint in every transaction.
%Ideally, we would like our quantum money to be publicly verifiable, that is,
%without resorting to the trusted authority for every interaction.
%Aaronson proved that public-key quantum money was possible relative to an oracle


