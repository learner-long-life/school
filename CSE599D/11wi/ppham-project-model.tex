\section{Quantum Money}

Consider notes which consist of a quantum state $\ket{\$_p}$ in
a fixed basis $\mathcal{B}$
and a classical serial number $p$,
together with a global classical function $A$ which computes
$p$ from the basis states. $\ket{\$_p}$ is a (possibly weighted) superposition
of basis states $\ket{G}$ which are equivalent in the sense that they all map
to the same value.

\begin{displaymath}
\ket{\$_p} = \frac{1}{\sqrt{N}} \sum_{G:A(G)=p} q_G \ket{G}
\end{displaymath}

With respect to some security parameter $\overline{D}$, 
we would like a different serial number $p$ to be produced each time
with probability exponentially close to 1 to prevent forgery through
repetition
of the minting algorithm. For each $p$, or even from each $\ket{G}$ where
$A(G)=p$, it should be difficult to find all equivalent basis states and
therefore
to forge their superposition. We will see later that it is useful to
shape the weights $q_G$ of the superposition, rather than have uniform
distribution. So how do we do this? Two words: knots, maybe.

\section{Knots}

Knots are like a loop of string which can be arbitrarily tangled
with
itself in three dimensions. We
can represent them in two-dimensions with $d \times d$
\emph{grid diagrams}, where strands
pass vertically and horizontally between $d$ \textsf{X} and
$d$ \textsf{O} markers,
one of each kind of marker in each column and row.
Equivalently, we can encode a grid diagram purely through
a pair of disjoint permutations $\Pi_{\textsf{X}}$
and $\Pi_{\textsf{O}}$ on the integers $\{1, \ldots, d\}$.

An Alexander polynomial can be computed for each knot based on its
crossings and is invariant under the Reidemeister moves. Therefore,
all grid diagrams of the same knot have the same Alexander polynomial
and can be transformed into one another.
However, it is conjectured to be hard (even for a quantum computer, on
average) to be able to generate all such equivalent
grid diagrams, or even more simply to determine if two grid diagrams are
equivalent. However, we can easily create a quantum
superposition of all grid encodings, compute their Alexander polynomials
$p$ into a second register, then measure $p$ to be left with the
superposition of all corresponding equivalent grid diagrams.
This one-wayness, due to quantum measurement, combined with the classical
one-wayness of digitally signing $p$, result in the one-wayness of
the minting algorithm.

Based on the leading notation above, you have probably guessed that the
basis $\mathcal{B}$ consists of all grid diagrams of size $d$ which ranges
from $2$ to $2\overline{D}$. The size of a grid diagram is
denoted as $d(G)$.