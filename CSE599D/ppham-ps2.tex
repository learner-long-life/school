\documentclass[12pt]{article}

\usepackage{fullpage}
\usepackage{fancyhdr}
\usepackage{cse-scribe}
\pagestyle{fancy}

\rhead{Problem \thesection\\Page \thepage\\Winter 2006}
\lhead{Paul Pham [ppham@cs.washington.edu]\\CSE 599D: Quantum Computing \\Problem Set 2}

%\renewcommand{\headrulewidth}{0.5pt}
\renewcommand{\footrulewidth}{0pt}

\addtolength{\headheight}{42pt} % make space for the rule
\addtolength{\headsep}{6pt} % make space for the rule

\renewcommand{\labelenumi}{\textbf{\alph{enumi})}}
\renewcommand{\labelenumii}{\textbf{\arabic{enumii}}}
%\renewcommand{\thesection}{\small{Problem \arabic{section}.}}
%  \makeatletter
%   \renewcommand{\section}{\@startsection{section}{1}{0mm}
%   {\baselineskip}%
%   {\baselineskip}{\normalfont\normalsize}}%
%   \makeatother
%\renewcommand{\section}{\@startsection{section}{1}}
\def\qopnamewl@#1{\mathop{\fam\z@#1}\nlimits@}
\def\Exp{\mathop{\rm {E}}}
\def\dist{{\rm dist}\,}

\input{Qcircuit}

\begin{document}
%\newcommand{\ket}[1]{|#1 \rangle}
%\newcommand{\bra}[1]{\langle #1 |}
\newcommand{\braket}[2]{\langle #1|#2 \rangle}
\newcommand{\normtwo}{\frac{1}{\sqrt{2}}}

%%%%%%%%%%%%%%%%%%%%%%%%%%%%%%%%%%%%%%%%%%%%%%%%%%%%%%%%%%%%%%%%%%%%%%%%%%%%%%%
%\setcounter{section}{0}
\section{Four-in-one Grover}

\begin{enumerate}

%------------------------------------------------------------------------------
\item % a
Each function is 1 on only one input and 0 on all others.
If you query on inputs 00, 01, and 10 and get a 1,
you can exactly determine whether you have $f_{00}$, $f_{01}$, or $f_{10}$.
If you get all zeros, then you know you have $f_{11}$.

%------------------------------------------------------------------------------
\item % b
Into the first two qubits, feed in the state:
\begin{displaymath}
\ket{\psi} = H^{\otimes 2}\ket{0} = \frac{1}{2}\sum_{x_3,x_4 \in \{0,1\}}\ket{x_3,x_4}
\end{displaymath}
Into the third qubit, feed the state $\ket{-}=\normtwo(\ket{0}-\ket{1})$.
Then we have:
\begin{eqnarray*}
U_{\alpha}(\ket{\psi}\otimes\ket{-}) & = &
\frac{1}{2}\sum_{x_1,x_2 \in \{0,1\}}\sum_{x_3,x_4 \in \{0,1\}}
\ket{x_1,x_2}\braket{x_1,x_2}{x_3,x_4} \otimes\\
& & \normtwo \sum_{y \in \{0,1\}}
\ket{y \oplus f_{\alpha}(x_1,x_2)}\bra{y}(\ket{0}-\ket{1})\\
& = & \frac{1}{2}\sum_{x_1,x_2 \in \{0,1\}}\ket{x_1,x_2} \otimes\\
& & \normtwo (\ket{f_{\alpha}(x_1,x_2)}\braket{0}{0} - \ket{\overline{f_{\alpha}(x_1,x_2)}}{\braket{1}{1}})\\
& = & \frac{1}{2}\sum_{x_1,x_2 \in \{0,1\}}(-1)^{f_\alpha(x_1,x_2)}\ket{x_1,x_2}
\end{eqnarray*}

%------------------------------------------------------------------------------
\item % c
In general, the inner product of any two states are:
\begin{eqnarray*}
& \braket{\alpha'}{\alpha} & = \frac{1}{4} \sum_{x_3,x_4 \in \{0,1\}}\sum_{x_1,x_2 \in \{0,1\}} (-1)^{f_{\alpha}(x_1,x_2)+f_{\alpha'}(x_3,x_4)}\braket{x_3,x_4}{x_1,x_2}\\
&                          & = \frac{1}{4} \sum_{x_1,x_2 \in \{0,1\}}(-1)^{f_{\alpha}(x_1,x_2)+f_{\alpha'}(x_1,x_2)}\braket{x_1,x_2}{x_1,x_2}\\
& & = \left\{ \begin{array}{lll}
\frac{1}{4} \sum_{x_1,x_2 \in \{0,1\}}1 & = 1 & \textrm{ if } \alpha = \alpha'\\
\frac{1}{4} (-1 + 1 + -1 + 1) & = 0 & \textrm{ if } \alpha \ne \alpha'
\end{array} \right.
\end{eqnarray*}

In the last line, we used the fact that two different functions $f_\alpha$
and $f_{\alpha'}$ will agree on two inputs, where they will be zero, and
disagree on the other two inputs, where one will be 1 and the other will be
0.

Therefore the inner product of a state with itself is 1, satisfying
normality, and with any other state is 0, satisfying orthogonality.

%------------------------------------------------------------------------------
\item % d
The following matrix transforms states in the  $\{\alpha\}$ basis to the
computational basis.
\begin{displaymath}
\frac{1}{2} \left[ \begin{array}{rrrr}
-1  &  1 &  1 &  1\\
 1  & -1 &  1 &  1\\
 1  &  1 & -1 &  1\\
 1  &  1 &  1 & -1
\end{array} \right]
\end{displaymath}

%------------------------------------------------------------------------------
\item % e
\begin{displaymath}
\Qcircuit @C=1em @R=.7em {
& \gate{H} & \ctrl{1} & \gate{H} \\
& \qw & \targ & \qw
}
\end{displaymath}
%In matrices:
%\begin{displaymath}
%H^{\otimes 2}(CNOT)H^{\otimes 2} = 
%\frac{1}{4}
%\left[ \begin{array}{rrrr}
% 1  &  1 &  1 &  1\\
% 1  & -1 &  1 &  -1\\
% 1  &  1 & -1 &  -1\\
% 1  &  -1 & -1 & 1
%\end{array} \right]
%\left[ \begin{array}{rrrr}
% 1  &  0 &  0 &  0\\
% 0  &  1 &  0 &  0\\
% 0  &  0 &  0 &  1\\
% 0  &  0 &  1 &  0 
%\end{array} \right]
%\left[ \begin{array}{rrrr}
% 1  &  1 &  1 &  1\\
% 1  & -1 &  1 & -1\\
% 1  &  1 & -1 & -1\\
% 1  & -1 &  -1 & 1
%\end{array} \right] =
%\left[ \begin{array}{rrrr}
% 1  &  0 &  0 &  0\\
% 0  &  0 &  0 &  1\\
% 0  &  0 &  1 &  0\\
% 0  &  1 &  0 &  0
%\end{array} \right]
%\end{displaymath}


\end{enumerate}

\pagebreak

%%%%%%%%%%%%%%%%%%%%%%%%%%%%%%%%%%%%%%%%%%%%%%%%%%%%%%%%%%%%%%%%%%%%%%%%%%%%%%%
\setcounter{section}{1}
\section{The Swap Test}

We denote the gate $H\otimes I \otimes I$ as $H'$ below.

\begin{enumerate}

%------------------------------------------------------------------------------
\item % a
\begin{eqnarray*}
\ket{x_1} & = H'\ket{0}\ket{\psi}\ket{\psi} & = \normtwo(\ket{0} + \ket{1})\ket{\psi}\ket{\psi}\\
\ket{x_2} & = C_{SWAP}\ket{x_1} & = \normtwo(\ket{0} + \ket{1})\ket{\psi}\ket{\psi}\\
\ket{x_3} & = H'\ket{x_2} & = \frac{1}{2}(\ket{0} + \ket{1} + \ket{0} - \ket{1})\ket{\psi}\ket{\psi} = \ket{0}\ket{\psi}\ket{\psi}
\end{eqnarray*}

The outcome of the measurement meter is $\ket{0}$ with 100\% probability.

%------------------------------------------------------------------------------
\item % b
\begin{eqnarray*}
\ket{x_1} = & H'\ket{0}\ket{\psi}\ket{\phi} & = \normtwo(\ket{0} + \ket{1})\ket{\psi}\ket{\phi}\\
\ket{x_2} = & C_{SWAP}\ket{x_1} & = \normtwo(\ket{0}\ket{\psi}\ket{\phi} + \ket{1}\ket{\phi}\ket{\psi})\\
\ket{x_3} = & H'\ket{x_2} & = \frac{1}{2}((\ket{0} + \ket{1})\ket{\psi}\ket{\phi} + (\ket{0} - \ket{1})\ket{\phi}\ket{\psi})\\
& & =
\frac{1}{2}(\ket{0}(\ket{\psi}\ket{\phi} + \ket{\phi}\ket{\psi}) + \ket{1}(\ket{\psi}\ket{\phi} - \ket{\phi}\ket{\psi}))\end{eqnarray*}
%
The outcome probability of the measurement meter is 50\% $\ket{0}$
and 50\% $\ket{1}$.

%------------------------------------------------------------------------------
\item % c
Create an entangled input state for the second and third qubits using
two orthogonal qubits $\braket{\phi}{\psi}$. Note that reversing the
roles of $\ket{\phi}$ and $\ket{\psi}$ introduces a $-1$ phase.
%
\begin{displaymath}
\frac{1}{2}(\ket{\psi}\ket{\phi}-\ket{\phi}\ket{\psi}) =
-\frac{1}{2}(\ket{\phi}\ket{\psi}-\ket{\psi}\ket{\phi})
\end{displaymath}
%
\begin{eqnarray*}
\ket{x_1} = \frac{1}{2}H'
\ket{0}(\ket{\psi}\ket{\phi} - \ket{\phi}\ket{\psi}) & = & \frac{1}{2\sqrt{2}}
(\ket{0}+\ket{1})(\ket{\psi}\ket{\phi}-\ket{\phi}\ket{\psi})\\
\ket{x_2} = C_{SWAP}\ket{x_1} & = & \frac{1}{2\sqrt{2}}(\ket{0}(\ket{\psi}\ket{\phi}-\ket{\phi}\ket{\psi}) + \ket{1}(\ket{\phi}\ket{\psi}-\ket{\psi}\ket{\phi})
)\\
\ket{x_3} = H'\ket{x_2} & = &
\frac{1}{4}\ket{0}(\ket{\psi}\ket{\phi} - \ket{\phi}\ket{\psi} +
\ket{\phi}\ket{\psi} + \ket{\psi}\ket{\phi}) +\\
& & \frac{1}{4}(\ket{1}(\ket{\psi}\ket{\phi} - \ket{\phi}\ket{\psi} - \ket{\phi}\ket{\psi} + \ket{\psi}\ket{\phi}))\\
& = &
\frac{1}{2}(\ket{1}(\ket{\psi}\ket{\phi} - \ket{\phi}\ket{\psi})
\end{eqnarray*}
%
The outcome of the measurement meter is $\ket{1}$ with 100\% probability.
This entangled state is orthogonal to every 2 qubit state of the
form $\ket{\psi'}\ket{\psi'}$:
%
\begin{displaymath}
\frac{1}{2}(\bra{\psi'}\bra{\psi'})(\ket{\psi}\ket{\phi} - \ket{\phi}\ket{\psi}) = \frac{1}{2}(\braket{\psi'}{\psi}\braket{\psi'}{\phi} - \braket{\psi'}{\phi}\braket{\psi'}{\psi}) = 0
\end{displaymath}

%------------------------------------------------------------------------------
\item %d
Given two $n$-qubit input states, perform the following operations:
entangle the two states using a NOT ($X^{\otimes n}$) gate and a Hadamard
($H^{\otimes n}$) gate. Feed this state into the SWAP test circuit of the
problem, modified to perform a $C_{SWAP}^{\otimes n}$) operation controlled
on the qubit $\ket{0}$.
If the states were identical, the measurement outcome will be $\ket{0}$,
and if the states were orthogonal, the measurement outcome will be
$\ket{1}$, with at least 50\% probability.
\end{enumerate}

\pagebreak

%%%%%%%%%%%%%%%%%%%%%%%%%%%%%%%%%%%%%%%%%%%%%%%%%%%%%%%%%%%%%%%%%%%%%%%%%%%%%%%
\section{A Continuous Time Search Problem}

\begin{enumerate}
%------------------------------------------------------------------------------
\item %a
We can form a state $\ket{\psi'}$ which consists of an equal superposition of
all states which are not $\ket{s}$.
\begin{eqnarray*}
\displaystyle
\ket{\psi'} & = & \frac{1}{\sqrt{2^n-1}}\sum_{x \in \{0,1\}^n, x \ne s}\ket{x}\\
\braket{s}{\psi'} & = & 0\\
\braket{\psi'}{\psi'} & = & 1\\
\ket{\psi'} & = & \frac{\sqrt{2^n}}{\sqrt{2^n-1}}(\ket{\psi} + \frac{1}{\sqrt{2^n}}\ket{s})
\end{eqnarray*}

%------------------------------------------------------------------------------
\item %b
\begin{eqnarray*}
& H & = \ket{s}\bra{s} + \ket{\psi}\bra{\psi}\\
&   & = \ket{s}\bra{s} + (\frac{1}{\sqrt{2^n}}\ket{s} + \frac{\sqrt{2^n-1}}{\sqrt{2^n}}\ket{\psi'})(\frac{1}{\sqrt{2^n}}\bra{s} + \frac{\sqrt{2^n-1}}{\sqrt{2^n}}\bra{\psi'})\\
&   & = \frac{2}{2^n}\ket{s}\bra{s} + \frac{\sqrt{2^n+1}}{2^n}(\ket{\psi'}\bra{s} + \ket{s}\bra{\psi'}) + \frac{2^n 2-1}{2^n}\ket{\psi'}\bra{\psi'}
\end{eqnarray*}
%------------------------------------------------------------------------------
\item %c
Because $H = I$ in the basis $\{\ket{s},\ket{\psi}$, $H^2 = I$, and
we have:

\begin{displaymath}
U = exp(-iHt) = (\cos{t})I + (i\sin{t})H
\end{displaymath}

%------------------------------------------------------------------------------
\item %d

\end{enumerate}

\end{document}
