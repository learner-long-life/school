\documentclass[12pt]{article}

\usepackage{fullpage}
\usepackage{fancyhdr}
\usepackage{scribe}
\pagestyle{fancy}

\rhead{Problem \thesection\\Page \thepage\\Autumn 2006}
\lhead{Paul Pham [ppham@cs.washington.edu]\\CSE 533: Error-Correcting Codes \\Problem Set 1}

%\renewcommand{\headrulewidth}{0.5pt}
\renewcommand{\footrulewidth}{0pt}

\addtolength{\headheight}{42pt} % make space for the rule
\addtolength{\headsep}{6pt} % make space for the rule

\renewcommand{\labelenumi}{\textbf{\alph{enumi})}}
\renewcommand{\labelenumii}{\textbf{\arabic{enumii}}}
%\renewcommand{\thesection}{\small{Problem \arabic{section}.}}
%  \makeatletter
%   \renewcommand{\section}{\@startsection{section}{1}{0mm}
%   {\baselineskip}%
%   {\baselineskip}{\normalfont\normalsize}}%
%   \makeatother
%\renewcommand{\section}{\@startsection{section}{1}}
\def\qopnamewl@#1{\mathop{\fam\z@#1}\nlimits@}
\def\Exp{\mathop{\rm {E}}}
\def\dist{{\rm dist}\,}

\begin{document}
\newcommand{\ket}[1]{|#1 \rangle}
\newcommand{\normtwo}{\frac{1}{\sqrt{2}}}

%%%%%%%%%%%%%%%%%%%%%%%%%%%%%%%%%%%%%%%%%%%%%%%%%%%%%%%%%%%%%%%%%%%%%%%%%%%%%%%
\setcounter{section}{0}
\section{Operations on Codes}

\begin{enumerate}

%------------------------------------------------------------------------------
% Part A
\item
Given an $(n,k,d)_q$ code, there exist two $n$-symbol codewords,
$c_1$ and $c_2$
which are separated by the minimum distance $d$. Consider two cases.

If we delete a symbol position in the code where $c_1$ and $c_2$ \emph{differ}
then the distance between them is decreased by one. In the worst case,
this was the only minimal distance in the original code, so now we have
decreased the minimum distance by one to $(d-1)$.

If we delete a symbol position in the code where $c_1$ and $c_2$ are the same,
the minimum distance does not change. If the distance changes for some other
pair of codewords, this degenerates to the first case.

Therefore we now have a new $(n-1,k,d'\ge d-1)_q$ code.

%------------------------------------------------------------------------------
% Part B
\item

%------------------------------------------------------------------------------
% Part C
\item

%------------------------------------------------------------------------------
% Part D
\item

%------------------------------------------------------------------------------
% Part E
\item

%------------------------------------------------------------------------------
% Part F
\item

\end{enumerate}

\pagebreak

%%%%%%%%%%%%%%%%%%%%%%%%%%%%%%%%%%%%%%%%%%%%%%%%%%%%%%%%%%%%%%%%%%%%%%%%%%%%%%%
% Problem 2
\setcounter{section}{1}
\section{Generalized Plotkin Bound}

\pagebreak

%%%%%%%%%%%%%%%%%%%%%%%%%%%%%%%%%%%%%%%%%%%%%%%%%%%%%%%%%%%%%%%%%%%%%%%%%%%%%%%
% Problem 3
\section{Bounds for Linear Codes}

\begin{enumerate}
%------------------------------------------------------------------------------
\item %a

%------------------------------------------------------------------------------
\item %b

%------------------------------------------------------------------------------
\item %c

\end{enumerate}

%%%%%%%%%%%%%%%%%%%%%%%%%%%%%%%%%%%%%%%%%%%%%%%%%%%%%%%%%%%%%%%%%%%%%%%%%%%%%%%
% Problem 4
\section{Binary Symmetric Channel}

\begin{enumerate}
%------------------------------------------------------------------------------
\item %a

%------------------------------------------------------------------------------
\item %b

%------------------------------------------------------------------------------
\item %c

\end{enumerate}

%%%%%%%%%%%%%%%%%%%%%%%%%%%%%%%%%%%%%%%%%%%%%%%%%%%%%%%%%%%%%%%%%%%%%%%%%%%%%%%
% Problem 5
\section{$q$-ary Erasure Channel}

\begin{enumerate}
%------------------------------------------------------------------------------
\item %a

%------------------------------------------------------------------------------
\item %b

%------------------------------------------------------------------------------
\item %c

%------------------------------------------------------------------------------
\item %d

\end{enumerate}

%%%%%%%%%%%%%%%%%%%%%%%%%%%%%%%%%%%%%%%%%%%%%%%%%%%%%%%%%%%%%%%%%%%%%%%%%%%%%%%
% Problem 6
\section{Tensor Product of Codes}

\begin{enumerate}
%------------------------------------------------------------------------------
\item %a

%------------------------------------------------------------------------------
\item %b

\end{enumerate}

\end{document}
