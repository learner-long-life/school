\documentclass[11pt]{article}
 \setlength{\textwidth}{6.5 in}
 \setlength{\textheight}{8.25in}
 \setlength{\oddsidemargin}{0in}
 \setlength{\topmargin}{0in}
 \addtolength{\textheight}{.8in}
 \addtolength{\voffset}{-.5in}

\usepackage{amssymb}
\usepackage{amsfonts}
\usepackage{amsmath}
\usepackage{latexsym}
\usepackage{epsfig}
\usepackage{bm}

\newcommand{\ignore}[1]{}
\newcommand{\R}{\mathbb{R}}
\newcommand{\Z}{\mathbb{Z}}
\newcommand{\C}{{\mathcal C}}
\newcommand{\D}{{\mathcal D}}
\renewcommand{\L}{{\mathcal L}}
\newcommand{\la}{\langle}
\newcommand{\ra}{\rangle}
\newcommand{\eps}{\epsilon}
\renewcommand{\P}{\mathrm{P}}
\newcommand{\E}{\mathbf{E}}
\newcommand{\NP}{\mathrm{NP}}
\newcommand{\Maj}{\mathrm{Maj}}
\renewcommand{\phi}{\varphi}
\newcommand{\bits}{\{-1,1\}}
\newcommand{\Inf}{\mathrm{Inf}}
\newcommand{\Stab}{\mathrm{Stab}}
\newcommand{\gap}{\mathrm{gap}}
\newcommand{\F}{{\mathbb F}}

\begin{document}
\noindent {\large {\bf CSE 533: Error-Correcting Codes}}  {\hfill {\bf Autumn 2006}}
\begin{center}
{\sc Problem Set 1} \\
{\bf Due on Monday, October 30} \\
\end{center}

\hline

\bigskip

\noindent {\bf Homework policy}: Students are encouraged to work on
the problems in small groups (of up to 3 people); however, all
writeups should be done individually and must clearly cite any
collaborators. You are strongly urged to try and solve the problems
without consulting any reference material besides what we cover in
class (such as any textbooks or material on the web where the solution
may appear either fully or in part). If for some reason you feel the
need to consult some source, please acknowledge the source and try to
explain what difficulty you couldn't overcome before consulting the
source and how it helped you overcome that difficulty. \\

\hline

\bigskip

\begin{enumerate}
\item {\bf Operations on Codes.} Prove the following statements (the
  notation $(n,k,d)_q$ code is used for general codes with $q^k$
  codewords where $k$ need not be an integer, whereas the notation
  $[n,k,d]_q$ code stands for a {\em linear code} of dimension $k$):
\begin{enumerate}
\item If there exists an $(n,k,d)_q$ code, then there also exists an $(n-1,k,d' \ge d-1)_q$ code. 
\item If there exists an $(n,k,d)_2$ code with $d$ odd, then there also exists an $(n+1,k,d+1)_2$ code.
\item If there exists an $(n,k,d)_{2^m}$ code, then there also exists an $(nm,km,d' \ge d)_2$ code.
\item If there exists an $[n,k,d]_{2^m}$ code, then there also exists an $[nm,km,d' \ge d]_2$ code.  
\item \label{item:part6} If there exists an $[n,k,d]_{q}$ code, then there also exists an $[n-d, k-1, d' \ge \lceil d/q \rceil]_q$ code.
\item If there exists an $[n,k_1,d_1]_q$ code and an $[n,k_2,d_2]_q$ code, then there exists an $[2n,k_1+k_2,\min (2d_1,d_2)]_q$ code.

\end{enumerate}


\item Prove the following version of the Plotkin bound for general alphabets. If $C \subseteq \Sigma^n$ is a (not necessarily linear) code over an alphabet of size $|\Sigma| =q$ with minimum distance $d > (1-1/q) n$, then $|C| \le \frac{qd}{qd-(q-1) n}$. 

\item 
\begin{enumerate}
\item Prove that for an $[n,k,d]_q$ linear code, we must have
\begin{equation}
\label{bd-1}
n \ge \sum_{i=0}^{k-1} \lceil d/q^i \rceil \ . 
\end{equation}
\underline{Hint}: Try induction together with Problem (\ref{item:part6})
above.
\item Deduce the Singeton bound for linear codes from bound (\ref{bd-1}) above.
\item Show that the simplex codes (dual of the Hamming codes) meet the bound (\ref{bd-1}). 
\end{enumerate}

\item \begin{enumerate}
\item Briefly argue (full proof not required) why the proof of
  Shannon's theorem for the binary symmetric channel 
that we did in class holds even if the encoding
  function $E$ is restricted to be linear.
\item Prove that for communication on ${\rm BSC}_p$, if an encoding
  function $E$ achieves a maximum decoding error probability (taken
  over all messages) that is exponentially small, i.e., at most
  $2^{-\gamma n}$ for some $\gamma > 0$, then there exists a $\delta =
  \delta(\gamma,p) > 0$ such that the code defined by $E$ has relative
  distance at least $\delta$.  In other words, good distance is {\em
  necessary} for exponentially small maximum decoding error
  probability.
\item Prove that if the encoding function $E$ is restricted to be
  linear, a similar conclusion holds even if only the {\em average}
  decoding error probability (computed for a message chosen uniformly
  at random) is exponentially small.
\end{enumerate}



\item 
\begin{enumerate}
\item For positive integers $k \le n$, show that less than a fraction
  $q^{k-n}$ of the $n \times k$ matrices $G$ over $\F_q$
  fail to generate a linear code of block length $n$ and dimension
  $k$. (Or equivalently, except with probability less than $q^{k-n}$, the rank of
  $G$ is $k$.)
\item 
Consider the $q$-ary erasure channel with erasure probaility
  $\alpha$ ($q\mbox{EC}_\alpha$, for some $\alpha$, $0 \le \alpha \le 1$): the input to this channel is a field element $x \in \F_q$,
  and the output is $x$ with probability $1-\alpha$, and an erasure
  '?' with probability $\alpha$.  For a linear code $C$ generated by an
  $n \times k$ matrix $G$ over
  $\F_q$, let $D : (\F_q \cup \{?\})^n \rightarrow C \cup
  \{\mathsf{fail}\}$ be the following decoder:
\[ D(y) = \left\{ \begin{array}{ll}
 c & \mbox{ if $y$ agrees with exactly one $c \in C$ on the unerased
 entries in $\F_q$} \\
\mathsf{fail} & \mbox{ otherwise} 
\end{array}
\right. \]
For a set $J \subseteq \{1,2,\dots,n\}$, let $P_{\rm err}(G|J)$ be the
probability (over the channel noise and choice of a random message) that $D$ outputs $\mathsf{fail}$ conditioned on the
erasures being indexed by $J$. Prove that the average value of $P_{\rm
  err}(G|J)$ taken over all $G \in \F_q^{n \times k}$ is less than
$q^{k-n+|J|}$. 
\item 
Let $P_{\rm err}(G)$ be the decoding error probability of the
  decoder $D$ for communication using the code generated by $G$ on the
  $q\mbox{EC}_\alpha$. Show that when $k = Rn$ for $R < 1-\alpha$, the
  average value of $P_{\rm err}(G)$ over all $n \times k$ matrices $G$
  over $\F_q$ is exponentially small in $n$.
\item Conclude that one can reliably communicate on the
  $q\mbox{EC}_\alpha$ at any rate less than $1 - \alpha$ using a
  linear code.
\end{enumerate}


\item We now define the tensor product operation that was used in
  class to discuss Elias' construction of iterated product of Hamming
  codes to achieve reliable communication at positive rate for ${\rm
  BSC}_p$ for some positive $p$ with an explicit construction and
  polynomial time decoding. We only focus on binary codes, but the
  definition applies over any field. Let $C_1$ be an $[n_1,k_1,d_1]_2$
  code with generator matrix $G_1 \in \{0,1\}^{n_1 \times k_1}$ and let $C_2$ be an $[n_2,k_2,d_2]_2$
  code with generator matrix $G_2 \in \{0,1\}^{n_2 \times k_2}$. The
  tensor product (or simply product) of $C_1$ and $C_2$, denoted $C_1
  \otimes C_2$, is a code of block length $n_1 n_2$ defined as
\[ C_1 \otimes C_2  = \{ G_1 M G_2^T \mid M \in \{0,1\}^{k_1 \times
  k_2}  \} \ . \]
\begin{enumerate}
\item Prove that $C_1 \otimes C_2$ is an $[n_1 n_2, k_1 k_2, d_1 d_2]$
  binary linear code.
\item Prove that codewords of $C_1 \otimes C_2$ correspond to
  $n_1 \times n_2$ matrices all of whose columns are codewords of
  $C_1$ and all of whose rows are codewords of $C_2$.
\end{enumerate}
\end{enumerate}
\end{document}