\documentclass{article}

\usepackage{eprint}

\begin{document}

\begin{center}
\LARGE{2010 NSF EAPSI Project Description}\\
\Large{Paul Pham}
\end{center}


\section{Host Researcher and Institution}

The EAPSI program would give me the valuable opportunity to understand the
international
nature of scientific collaboration by allowing me to form new research
and personal connections. If awarded this
fellowship, I would travel to Australia for the first time, which is one of
the world's centers for my chosen field of quantum computing. Australia is home
to Engineered Quantum Systems (EQuS), a multi-institution, international
collaboration for engineering quantum systems and technology.
In addition, one of the EQuS nodes is especially well-suited to my
collaboration, the University of Sydney.
The Quantum Science Research Group there contains strong researchers 
including Drs. Stephen Bartlett and Andrew Doherty on the
side of theory as well as Drs. Michael Biercuk and David Reilly on the side of
experiment, using trapped ions.
Ion traps have an advantage over
other qubit candidates in possessing relatively long coherence times,
identical energy characteristics from qubit to qubit, and a recent
demonstration of a fully-programmable two-qubit trapped ion quantum processor
 \cite{Hanneke2009}.

Dr. Michael J. Biercuk would be an ideal host due to his past experience at NIST
Boulder in achieving high-fidelity quantum control over trapped ions
\cite{BUVSIB2009a}.
In particular, he has developed advanced experimental techniques for 
suppressing noise in trapped ions using a technique called dynamical
decoupling (DD) which has increased the reliability of initial measurements
by several orders of magnitude.
Dr. Biercuk is further pursuing this research
at the University of Sydney where he is a lecturer and head of the newly-established
Quantum Control Laboratory, where I propose to carry out my fellowship in
dealing with the central problem of suppressing quantum noise.

\section{Quantum Noise}

In order to realize the benefits of quantum information processing and
to make a quantum computer into a reality, we need to deal with the
practical problems of fault-tolerance and noise.
Any quantum system must be shielded from
unintended couplings to its environment in order to maintain a coherent 
quantum state that is useful for computation. These stray couplings are
known as noise, errors, or {\em decoherence}. Remarkable theoretical results in
{\em fault-tolerant} quantum computing have shown that we can correct these
seemingly continuous errors using discrete operations, unlike classical
analog computation. Most current approaches to fault-tolerance focus on
quantum error correcting (QEC) codes, which encode
each logical qubit into many physical qubits. However, every QEC code introduces
an overhead of more physical qubits, which increases the probability of
failure. We need to achieve a "fault-tolerant threshold" for a single qubit's
error probability in order to ensure we get a net benefit from applying QEC.
Threshold theorems for a variety of architectures have proven that this single
qubit error is anywhere from $10^{-3}$ to $10^{-6}$ \cite{NC2000}.
If $p$ is the single qubit error and $p_{th}$ is the threshold error rate,
the error rate for the encoded logical qubit is $p(p/p_{th})^{2^k}$ \cite{NC2000}.
Therefore, to fully realize the benefits of driving down error rates with
concatenating QEC codes, we need the ratio $p/p_{th}$ to be as small as possible,
and therefore our goals for $p$ are much more stringent than for unencoded
qubits.

\section{Dynamical Decoupling}

Dynamical decoupling is a promising approach for achieving this much lower error
rate without introducing more qubits or measurement---only precise, unitary quantum
control are needed via programmed pulses.
Based on open-loop control techniques pioneered with
nuclear magnetic resonance (NMR), dynamical decoupling uses sequences of
precisely timed pulses to perform quantum bit flips to coherently average out
fluctuations from the environment \cite{VL1998}.
The most famous example of a dynamical
process for suppressing noise is Hahn's spin-echo sequence, which was
developed in NMR to remote arbitrary dephasing due to inhomogeneities in the
magnetic field \cite{}. Dr. Biercuk was able to use a similar technique
to increase phase coherence times in trapped beryllium ions from 2.5ms to 700ms, an
increase of {\em two orders of magnitude} \cite{BUVSIB2009a}.
Unfortunately, the NMR community has long known about several limitations
to a simple DD approach, using, for example a repeated sequence of
pulses attributed to Carr, Purcell, Meiboom, and Gill (CPMG). The CPMG
sequence is only effective against slowly varying phase noise, whereas
in a general quantum information setting, we may experience phase noise
of arbitrary spectal density.

Recent theoretical results indicate that more general kinds of noise can be further
suppressed with longer and more complicated pulse sequences called
Uhrig Dynamical Decoupling (UDD) \cite{Uhrig2007}, where the delays between various
pulses in a sequence can be varied precisely.
Until recently, experimental errors were too great to allow the precise
implementation of these new sequences. However, Dr. Biercuk's group at NIST
were able to overcome these limitations by achieving high operational
and measurement fidelities and engineering the noise
environment itself \cite{BUVSIB2009a}, and thus were able to achieve
the first ever experimental demonstration of UDD and a corresponding
suppression of errors by 8 to 10 orders of magnitude for some kinds of
high-frequency noise.

Dynamical decoupling can be useful for "refreshing" arbitrary quantum states
in a quantum memory. This provides much needed volatile storage in between
computational operations, similar to a DRAM for classical computers. In the
future, this research could give rise to a "quantum firmware" which will
provide a platform for higher-level quantum computation.
Dr. Biercuk was also able to use experimental feedback to
generate optimized sequences for such a model quantum memory \cite{BUVSIB2009c}.

Further theoretical frameworks have been proposed to extend these results,
most importantly Dynamical Controlled Gates (DCGs) to perform non-trivial gates
\cite{Khodjasteh2009},
other than just the "no-op" of preserving quantum states.
This is crucial for performing real quantum computations since even a simple
logical gate in a quantum algorithm may result in many physical gates
\cite{Dawson2005}.
Even small errors in enacting individual gates may accumulate and render
the overall computation useless, unless additional suppression is done.

\section{Project Description: Pulse Programming}

In order to generate these complex sequences of programmable pulses,
high-speed digital electronic hardware is needed as well as a flexible software
interface. Because Dr. Biercuk is starting a new lab, he will have the
opportunity to redesign and improve the pulse programming system at NIST
to validate the theoretical proposals above and further extend his
experimental results. The system will need to be able to generate
transistor-to-transistor logic levels (TTL) as well as generate radiofrequency
(RF) pulses of a programmable phase, frequency, and amplitude using
direct digital synthesis (DDS) in 8 independent, synchronized channels.
The system will also need to generate arbitrary analog waveforms using
a digital-to-analog converter (DAC).
Finally, the system will need to be able to count input pulses from a
photomultiplier tube (PMT) and perform conditional actions based on these
counts, such as branching on whether a threshold has been reached in order to
perform state detection. Ideally this system would be constructed from
open source and/or commodity off-the-shelf components in order to decrease
its cost and increase its reusability.

The proposed project will include designing, assembling, and testing
custom circuit boards, integrating them with commercial boards, and
writing control software to perform the above tasks. It will make heavy
use of reusable, open source components that have already been tested in other
ion trap experiments, as described in \cite{pp2005}.
Thus, this project will form one part of a general-purpose quantum control toolkit,
which is the larger goal of Dr. Biercuk's laboratory.

\section{Applicant}

I would be a valuable contributor to this effort due to my past experience in
designing and implementing electronic pulse programming systems to control
ion traps \cite{Pham2005} \cite{Petersen2007} \cite{Schindler2008}.
I have a solid track record of openly contributing
to the research community for over five years, both in terms of personally
engineering systems in two European laboratories \cite{Riebe2006}
 and one laboratory in the U.S.,
as well as maintaining an open source project on the internet
\cite{pp2005}
from which other
researchers can learn and to which they have contributed. I have
been cited by the company MagiQ in their successful
Small Business Innovation Research (SBIR)
grant for pulse programming systems \cite{MagiQ2007}
and have also been retained as a consultant for the same company.

I
frequently correspond with professors, post-docs, and grad students from groups
in Germany, Denmark, England, Austria, and the across the U.S. in my quest to
better understand the scientific process. It is my belief that we are limited
more by social and political constraints rather than technological or physical
limitations. My work so far in this field is part of a
new trend of applying reconfigurable computing and open source
methodologies to physics research, as evidenced by NIST Boulder's decision
to release their ion trap control software as open source \cite{ionizer2010}.

I am passionate about open discussion and sharing, so successful in
the software world and internet development, and would love the opportunity to
bring this perspective and my experience to help Dr. Biercuk in his research,
thereby bringing us closer to a functioning quantum computer.

\bibliography{ppham-project-description}
\bibliographystyle{tocplain}

\end{document}