% Paul Pham's Bio Sketch for EAPSI Fellowship

\documentclass[letter]{article}
\usepackage{fancyhdr}
\usepackage[pdftex]{graphicx,color,hyperref}

\pagestyle{fancyplain}
\lhead[\fancyplain{}{\bfseries\thepage}]
    {\fancyplain{}{\bfseries\rightmark}}
\rhead[\fancyplain{}{\bfseries\leftmark}]
    {\fancyplain{}{Paul T. Pham \hfill\bfseries\thepage}}
\cfoot[]{}
\addtolength{\headheight}{1.6pt}

\addtolength{\topmargin}{-0.75in}	% repairing LaTeX's huge margins...
\addtolength{\textwidth}{.5in}	% same here....
\addtolength{\marginparwidth}{.1in}	% The word ``Memberships'' is really long below
\setlength{\headwidth}{\textwidth}
\setlength{\parindent}{0mm}	% indented paragraphs just don't make it
			   	% on a resume.  (My opinion; if you
			   	% disagree, give this whatever value you
			   	% want.) 
\setlength{\textheight}{9.5in}	% more margin hacking.  I have a large
			    	% resume to fit on one page.

\begin{document}

\thispagestyle{empty}           % no headers on the first page

\reversemarginpar		% this puts the margin notes on the
				% left-hand side of the resume.

{\LARGE {2010 NSF EAPSI Biographical Sketch}}
\par
{\Large {\bf Paul T. Pham}}
\par
\vspace{.25in}
E-mail: ppham@cs.washington.edu
\hspace*{\fill}
205 13th Ave E, \#5
\linebreak
Phone: (206) 859-0322
\hspace*{\fill}
Seattle, WA 98102

				% address stuff.  The \hspace*{\fill} is
				% necessary to get things laid out
				% correctly.  The sizes fed to \vspace are
				% variable, depending on how much
				% whitespace you want, and how much
				% stuff you're trying to fit on a page....
\par
\vspace{.25in}
				% if the \marginpar is at the beginning
				% of the line, it loses.  *sigh*.  This
				% is a typical entry in a resume.   (I'm
				% defining an entry as something
				% important enough to have a marginal note
				% setting it off.)  Give it some space
				% with \vspace, use \par to break
				% between paragraphs, enter your text,
				% and place the \marginpar at the end.
				
%Yes\marginpar{{\bf U.S. Citizenship}}

{\bf Pulse Programmer} \hspace*{\fill}January 2005---Present
\marginpar{{\bf Open Source}}
\par
\url{http://pulse-programmer.org}
\marginpar{{\bf Projects}}
\par
Project Founder, Lead Developer
\par
An open platform for general-purpose, programmable radiofrequency systems.
\vspace{\baselineskip}
\par

{\bf Quantum Compiler} \hspace*{\fill}July 2005---Present
\par
\url{http://quantum-compiler.org}
\par
Project Founder
\par
An open implementation of compiling algorithms to approximate arbitrary quantum
logic gates.
\par
\vspace{\baselineskip}
\par

{\bf University of Washington} \hspace*{\fill}September 2005---Present
\marginpar{{\bf Education}}
\par
Degree in progress: Doctor of Philosophy in Computer Science.
Advisor: Dave Bacon
\vspace{\baselineskip}
\par

{\bf Massachusetts Institute of Technology}
\par
Master of Engineering in Electrical Engineering \& Computer Science, February 2005.\\
Graduate GPA: 4.7/5.0\\
Thesis: \href{http://sourceforge.net/project/showfiles.php?group_id=129764&package_id=144780&release_id=307201}{A general-purpose pulse sequencer for quantum computing}.\\
Advisor: Isaac Chuang\\
\par
Bachelor of Science in Electrical Engineering and Computer Science, June 2004.\\
Undergraduate GPA: 4.2/5.0
\vspace{\baselineskip}
\par
				% This is a slightly more complicated
				% example.  \vspace gets a `doublespace'
				% (1.5\baselineskip) between entries, and
				% a  `singlespace' between paragraphs of
				% an entry.
				% (1.5\baselineskip means twice the value
				% of \baselineskip).  

\vspace{\baselineskip}
\par
{\bf {University of Washington Dept. of Physics and Astronomy}} \hfill Seattle, WA
\marginpar{{\bf Research}}
\par
{\em Lab Technician} \hfill January---July 2007
\marginpar{{\bf Experience}}
\par
\href{http://depts.washington.edu/qcomp/}{Trapped Ion Quantum Computing Group}
\par
Supervisor: Prof. Boris Blinov
\par
Built a programmable radio-frequency system for ion trap laser control. Designed and
assembled high-speed printed circuit boards. Developed FPGA firmware and
software for Ethernet control
and an instruction set architecture for generating fast pulses.\\
\url{http://pulse-sequencer.sf.net}

\par

\vspace{\baselineskip}
\par
{\bf Max Planck Institute for Quantum Optics} \hfill Garching, Germany
\par
{\em Visiting Ph.D. Student} \hfill July 2005---August 2005
\par
\href{http://www.mpq.mpg.de/qsim/}{Quantum Analog Simulation Group}
\par
Advisor: Dr. Tobias Sch\"atz
\par
Built a programmable radio-frequency system for ion trap laser control.
\par

\vspace{\baselineskip}
\par
{\bf University of Innsbruck} \hfill Innsbruck, Austria
\par
{\em Visiting Ph.D. Student} \hfill February 2005---June 2005
\par
\href{http://heart-c704.uibk.ac.at/index.html}{Quantum Optics and Spectroscopy Group}
\par
Advisor: Univ. Prof. Rainer Blatt
\par
Built a programmable radio-frequency system for ion trap laser control.
\par

\vspace{\baselineskip}
\par
{\bf MIT Center for Bits and Atoms} \hfill Cambridge, Massachusetts
\par
\href{http://web.mit.edu/~cua/www/quanta/}{quanta Research Group}
\par
{\em Graduate Research Assistant} \hfill September 2003---January 2005
\par
Advisor: Prof. Isaac Chuang
\par
Designed and built instrumentation for quantum computing experiments.
\par

\vspace{\baselineskip}
\par
``Component-based Invisible Computing.''
\marginpar{{\bf Publications}}\\
A. Forin, J. Helander, \textbf{P. Pham}, J. Rajendiran.\\
IEEE Realtime Embedded Systems Workshop Paper, December 2001.

\newpage

``Method and system for managing the execution of threads and data processing.''
\marginpar{{\bf Patents}}\\
A. Forin, J. Helander, \textbf{P. Pham}.\\
U.S. Patent Application No. 20030233392.\\
Filed on June 12, 2002.

\vspace{\baselineskip}
University of Washington Max E. Gellert Fellowship (2005)\marginpar{{\bf Awards}}\\
MIT Jerome B. Wiesner Scholarship (1999-2000)\\
Robert C. Byrd Scholar (1999-2001)\\
National Merit Scholar (1999)\\
Oklahoma All-State Scholar (1999)\\
Microsoft Computer Science Scholarship (1999)
\par

\vspace{\baselineskip}

{\bf University of Washington} \hfill Seattle, Washington
\marginpar{{\bf Teaching}}

\par
{\em Teaching Assistant, Computer Science \& Engineering Department}
\marginpar{{\bf Experience}}

\vspace{0.5\baselineskip}
\par
The Hardware/Software Interface (\href{http://www.cs.washington.edu/education/courses/351/10sp/}{CSE 351})\hfill Spring 2010
\par
Professor Gaetano Borriello

\vspace{0.5\baselineskip}
\par
Data Structures (\href{http://www.cs.washington.edu/education/courses/326/06au/}{CSE 326})\hfill Autumn 2006
\par
Professor Larry Snyder

\vspace{0.5\baselineskip}
\par
Software Development Tools (\href{http://www.cs.washington.edu/education/courses/303/06sp/}{CSE 303})\hfill Spring 2006
\par
Professor Magda Balazinska

\vspace{0.5\baselineskip}
\par
Algorithms (\href{http://www.cs.washington.edu/education/courses/417/06wi/}{CSE 417})\hfill Winter 2006
\par
Professor Larry Ruzzo

\vspace{0.5\baselineskip}
\par
Discrete Structures Class (\href{http://www.cs.washington.edu/education/courses/321/05au/}{CSE 321})\hfill Autumn 2005
\par
Professors Dieter Fox \& Anna Karlin
\par

\vspace{\baselineskip}
\par
{\bf MIT Elec. Eng. \& Computer Science Dept.} \hfill Cambridge, Massachusetts

\vspace{0.5\baselineskip}
\par
{\em Teaching Assistant}
\par
Software Engineering Laboratory Class (\href{http://courses.csail.mit.edu/6.170/old-www/2004-Spring/admin-info/generalinfo.html#Staff}{6.170}) \hfill January 2004---May 2004
\par
Professor Rob Miller

\vspace{0.5\baselineskip}
\par
{\em Lab Assistant}
\par
Software Engineering Laboratory Class (6.170) \hfill September 2002---May 2003
\par
Professors Michael Ernst \& Daniel Jackson
\par

\end{document}




