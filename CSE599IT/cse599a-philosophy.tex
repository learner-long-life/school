\documentclass{article}

\usepackage{fullpage}

\begin{document}

\begin{center}
\Large{CSE 599A: Statement of Teaching Philosophy}\\
\vspace{0.4\baselineskip}
\large{Paul T. Pham}
\end{center}

In observing my academic role models, I have learned that pure
research ability is necessary to be a successful professor but not
sufficient.  Teaching ability is strongly correlated with research
ability, since it involves effectively communicating with colleagues,
students (who may be future colleagues), funding sponsors,
administrators, researchers in other disciplines, and the general
public. As a teaching assistant, I gained a deeper understanding of
the material by applying my research experience to designing homeworks
and lectures.  As a student myself, I learned to draw connections
between classroom lectures, current publications, and open problems in
the field.  As a current and future researcher, I intend to teach as a
way of promoting my field, focusing my work, and attracting new
students, whom I will also encourage to teach.

In my graduate career and beyond, I will evaluate and commit to
students as an investment in my future research environment,
combatting a counterproductive monoculture in computer science. There
were few women or minority professors in my department at MIT,
resulting from attrition of already small graduate and undergraduate
populations.  As a teacher, I will encourage all of my students,
especially those who were underrepresented, to continue in the field.

However, I found one discrepancy particularly worrisome: I have met only one
other Vietnamese student in computer science and no Vietnamese faculty
members at all in my five years of university education.  I have often
felt that my career goals are pulling me further into the traditions
of Western science and away from my cultural heritage.  At the same
time, I know that there are several rising technology universities in
Vietnam which are largely isolated from the international research
community. Without outreach efforts from immigrants overseas, these
schools will be overrun by the agendas of the Vietnamese
government and American businesses.
I fear an education in those
circumstances would make young people in Vietnam less free by making
them more valuable tools of the Communist regime, from which my parents
fled after the war and which prevents them from returning.
Therefore, I believe one of my most important roles as a teacher is to
seek out leadership roles where I can help sponsor
Vietnamese students to study abroad and to advance their research
careers in both industry and academia. It is likely that such students
will be required to return to Vietnam after their studies to preserve
the country's intellectual resources. While I believe this is an
admirable goal, I hope some day this choice will be voluntary. Furthermore,
I hope some day to return myself to a free and democratic Vietnam as a
teacher.

I believe the role of a teacher is to provide a context for knowledge,
connecting with a previous framework, applications, industry, and other
disciplines. To become an expert only takes an infinite amount of effort and
volumes of textbooks; however, this approach would drive away all but the
most zealous students. The advantage of a human expert is to select, from
his or her experience, the essence of concepts and not the artifacts of their
expressions.
This knowledge is then presented to students as something
embedded in the life of another human, and encourages to make this knowledge
their own through solving problems and making mistakes.

Teachers prioritize knowledge and provide feedback, and in this
way guide students along a particular intellectual path. After a point,
the distinction between teacher and student should be blurred; colleagues
should feel free to split paths or rejoin them later. Sometimes it will
make more sense to travel together; other times collaborators will have
different destinations.
 In my case, I
believe that my paths should be joined with research. I will
encourage my students to evaluate me, and their other teachers,
as carefully as teachers evaluate students,
because critical judgment is a key scientific skill.
Science does not occur in a vacuum; as a human enterprise
it is necessarily social, and even the best research does no good if it
languishes in obscurity. The dual of research then is its dissemination
in a way that is easy to acquire and retain: this is exactly what I
aspire to do when I teach.


\end{document}