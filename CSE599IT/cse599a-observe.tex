\documentclass{article}

\usepackage{fullpage}

\begin{document}

\begin{center}
\Large{CSE 599A: Being Observed as a TA}\\
\vspace{0.4\baselineskip}
\large{Paul T. Pham}
\end{center}

For the autumn quarter of 2005, I was a teaching assistant for two quiz
sections of the discrete structures class, CSE 321. My 12:30pm section
was observed by Kayur Patel on Thursday, November 10th, and my 1:30pm
section was observed by Valentin Razmov on Thursday, November 17th.
Although I was comfortable with the presence of outside observers,
I discovered that they affected my teaching style in different ways. The
feedback I received from each observer also highlighted the differences
in the collective personalities of my two sections, which I had
previously underestimated.

In my earlier section, Kayur noticed that my students did not participate
as much in classroom discussion; I often had to wait a noticeable time
for volunteers, and students were not quick to ask questions, request
elaboration, or suggest corrections. He noticed that I was comfortable
speaking in a classroom setting from my previous teaching experience,
but also that I spent a (perhaps too long) amount of time providing
motivation and drawing connections to other fields compared to covering
concrete problems. Part of this stemmed from the fact that I knew
Kayur would be bored in my section, and I wanted to provide some
general-interest material for both him and my students.

After discussion with Kayur, I realized that my
early section, although perhaps shy, also expected me to be a distant
giver of knowledge and my section to be a passive activity. No one ever
objected to mistakes which I discovered in later review, and my humor
was accepted with indifference or as an extraneous sugar coating to
knowledge. I made the mistake
of calling on students involuntarily to answer questions or come to the board
in an effort to involve them more. This clashed with their expectations of me
and unfortunately alienated them further. I was disappointed to discover
in the mid-term course evaluation by CIDR that most of these students
found me unhelpful.

In my later section, Valentin observed that not only was I relaxed and
able to use humor, but that my students were also relaxed and encouraged
to participate more. Students in this section often volunteered answers
to my questions and questions of their own without my prompting. They
expected me to be a discussion leader and not a lecturer, so they were
quick to point out my mistakes or request topics that differed from my
agenda. I also found that I was more careful to explain things clearly and
in multiple ways because Valentin was in the audience. After discussion with
Valentin, I became conscious of the improvements I made to my teaching
style for my second section as a result of mistakes in my first section.
I often modified activities that did not work, shortened or clarified
explanations, and was more prepared to answer questions in the later
section. This corresponded with a more positive CIDR evaluation from
these students.

The teaching seminar, CSE 599A, gave me valuable feedback on my
effectiveness in the classroom during term. This allowed me to implement
improvements and act on this feedback while it still mattered. As any
teacher knows, feedback is more valuable than gold because it is
difficult to see one's self from a third-person perspective.
Moreover, the instructors of the course were too busy to sit in on
quiz sections, and I felt I was too busy to request a CIDR observer.
Therefore, the teaching seminar provided my only opportunity to
evaluate my teaching skills this quarter.


\end{document}