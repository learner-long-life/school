\mode<article>
{
  \usepackage{fullpage}
  \usepackage{hyperref}
  \setjobnamebeamerversion{qdb-talk.beamer}
}

\mode<presentation>
{
  \usetheme[compress]{Dresden}

  \setbeamercovered{transparent}
}

\usepackage{graphicx}
\usepackage[latin1]{inputenc}
\usepackage[english]{babel}
\usepackage{xypic}

\title{Quantum Algorithms for Relational Databases}
\author{Paul~Pham}
\date{2 June 2006}
\subject{}

\institute[CSE 544]{
  CSE 544\\
  Quantum Databases}

\input{Qcircuit}

\begin{document}

\frame{\maketitle}

%%%%%%%%%%%%%%%%%%%%%%%%%%%%%%%%%%%%%%%%%%%%%%%%%%%%%%%%%%%%%%%%%%%%%%%%%%%%%%%
\section{Overview}

%------------------------------------------------------------------------------
\frame[label=overview]{

  \frametitle{Overview}

  \begin{itemize}
  \item Crash course in quantum computing.
  \item Grover's search algorithm.
  \item Quantum RAM
  \item Algorithms for a classical RDBMS.
  \item Future work
  \end{itemize}
}

%%%%%%%%%%%%%%%%%%%%%%%%%%%%%%%%%%%%%%%%%%%%%%%%%%%%%%%%%%%%%%%%%%%%%%%%%%%%%%%
\section{Quantum}

%------------------------------------------------------------------------------
\frame[label=qubit-intro] {

  \frametitle{One Qubit, Two Qubit, \ldots}

  Quantum computing represents information in a two-state subsystem
  called a \textbf{qubit}.

  \vspace{0.5\baselineskip}

  A classical bit is either 0 or 1.
  
  A qubit $\ket{\phi}$ can be a mix
  $\alpha\ket{0} + \beta\ket{1}$.

  \vspace{0.5\baselineskip}

  $\ket{\phi}$ is a 1-qubit state. $\ket{0}$ and $\ket{1}$ are
  \textbf{basis states} which we interact with classically.
  When we measure $\ket{\phi}$, it ``collapses'' to either $\ket{0}$ with
  probability $|\alpha|^2$ or $\ket{1}$ with probability $|\beta|^2$
  ($|\alpha|^2 + |\beta|^2 = 1$).

  \vspace{0.5\baselineskip}

  $n$ qubits combine to form a mix of $2^n$ basis states.

}

%------------------------------------------------------------------------------
\frame[label=qubit-tensor] {

  \frametitle{$\ldots$ $n$ Qubits}

  An \textbf{exponential} number of states in \textbf{linear}
  time and space.

  \footnotesize
  \begin{center}
    \begin{displaymath}
    \xymatrix @R=0.3em {
     &            & &            & & \ket{000}\\
     &            & &            & & \ket{001}\\
     &            & & \ket{00} & & \ket{010}\\
     \ket{\phi_1} \ar[r] \ar[dr]
                 & \ket{0} &
     \ket{\phi_1,\phi_2} \ar[r] \ar[ur] \ar[dr] \ar[ddr]
                 & \ket{01} &
     \ket{\phi_1,\phi_2,\phi_3} \ar[r] \ar[ur] \ar[uur] \ar[uuur]
                                  \ar[dr] \ar[ddr] \ar[dddr] \ar[ddddr]
                 & \ket{011}\\
      & \ket{1} & & \ket{10} & & \ket{100}\\
      &           & & \ket{11} & & \ket{101}\\
      &         & &  & & \ket{110}\\
      &         & &  & & \ket{111}
     }
  \end{displaymath}
  \end{center}
  \normalsize

}

%%%%%%%%%%%%%%%%%%%%%%%%%%%%%%%%%%%%%%%%%%%%%%%%%%%%%%%%%%%%%%%%%%%%%%%%%%%%%%%
\section{Grover}

%------------------------------------------------------------------------------
\frame[label=grover-intro]{

  \frametitle{Grover Finds a Way}

  You are given:
  \begin{itemize}
    \item $N=2^n$ items $x_1,\ldots,x_N$
    \item A black-box oracle function $f:\{x_i\}\rightarrow\{0,1\}$
    \item A promise that there is a single marked stated $f(x_m)=1$.
  \end{itemize}

  How do you find $x_m$?
  \begin{itemize}
    \item Classically, you must query at least $N/2$ items randomly.
    \item Quantumly, you can get away with $\sqrt{N}$ queries [Gro96]
  \end{itemize}

  In fact, \textbf{this is optimal}. [Amb99]
  In general, finding $t$ marked items takes $O(\sqrt{Nt})$ running time.

  \vspace{0.5\baselineskip}

  Why query complexity?
  Most oracles $f$ take at least as much classical time as quantum time.
}

%------------------------------------------------------------------------------
\frame[label=grover-iterate]{

  \frametitle{Grover Search as Iterated Rotation}

  Let's call the mix of $t$
  \textbf{marked states} $\ket{\alpha}$
  and the mix of the $N-t$ unmarked states as $\ket{\beta}$.

  \begin{enumerate}
    \item Initialize your $n$-qubit register to the equal mix state
          $\ket{\psi}_n$.
    \item Repeat the following $\frac{\pi}{4}\sqrt{N/t}$ times:
    \begin{enumerate}
      \item Evaluate the query function $f$ on the components of the
            current state $\ket{\psi'}$ in parallel.
      \item Flip the marked states about the unmarked states.
      \item Flip all states about the original $\ket{\psi}_n$.
    \end{enumerate}
    \item Measure the register.
    \item $\ldots$
    \item Profit! (Get the right answer with high probability).
  \end{enumerate}

  \vspace{0.5\baselineskip}

  \begin{itemize}
    \item Two flips = rotation in the subspace spanned by $\ket{\alpha}$
          and $\ket{\beta}$.
    \item We are rotating closer to the marked states
          (maximizing their probability of measurement).
  \end{itemize}
}

%------------------------------------------------------------------------------
\frame[label=grover-]{

  \frametitle{Unstructured Search Problems}

%  Grover's search assumes we know the number of answers $t$.
%  We can count solutions without enumerating them
%  using \textbf{quantum counting}, also in $O(\sqrt{N})$ [BHT96].

%  \vspace{0.5\baselineskip}
  
  What are some unstructured problems?
  \begin{itemize}
    \item 3SAT, NP-complete problems (over Boolean ass'ts to $n$ vars)
    \item Cryptographic key search (over all $n$-bit keys)
    \item Hard 4-coloring instances, other brute-force problems.
  \end{itemize}

  \vspace{0.5\baselineskip}

  In reality, most databases are ordered and use indexes.
  Quantum search still gives a speedup, but only by a constant factor
  ($0.531\log{n}$) [FGGS99].

  \vspace{0.5\baselineskip}

  This is not exciting enough for a course project, so we need to find
  problems where classical structure does not help.
}

%%%%%%%%%%%%%%%%%%%%%%%%%%%%%%%%%%%%%%%%%%%%%%%%%%%%%%%%%%%%%%%%%%%%%%%%%%%%%%%
\section{QRAM}

%------------------------------------------------------------------------------
\frame[label=qram-cram]{

  \frametitle{Arbitrary Relations and Classical RAM}

  \begin{center}
     \includegraphics[height=0.75in]{cram}
  \end{center}

  \begin{itemize}
    \item In reality, databases do not contain every possible tuple.
    \item How do we store arbitrary relations classically?
    \item A RAM with $a$-bit addresses and $d$-bit data words
          can store an arbitrary relation on bits:
    \begin{itemize}
      \item Up to arity $d$
      \item Up to $2^a$ tuples.
    \end{itemize}
    \item To get efficient algorithms, we assume constant-time access.
  \end{itemize}
}

%------------------------------------------------------------------------------
\frame[label=qram-hybrid]{

  \frametitle{Quantum-Classical Hybrid Model}

  Any practical model must combine classical and quantum computation.

  \vspace{0.5\baselineskip}

  \begin{center}
  \begin{tabular}{|l|c|l|}
  \textbf{Classical Computer} & $\Rightarrow $ & \textbf{Quantum Computer}\\
   & Program description &\\
  Deterministic control &  & Quantum operations\\
  User interfaces       &  &\\
  Hard drive storage    & Measurement values  & \\
  Normal binary states  & $\Leftarrow $ & Quantum state
  \end{tabular}
  \end{center}
}

%------------------------------------------------------------------------------
\frame[label=qram-circuit]{

  \frametitle{Quantum Random Access Memory (QRAM)}

  A quantum RAM with $a$-bit addresses and $d$-bit data
  words has classical lines for reading and writing and
  quantum lines for the search algorithm.

  Classical data is encoded into qubits. Addresses can mixed,
  and the output is mixed data associated (\textbf{entangled}) with
  the addresses.

  \begin{center}
  \footnotesize
  \begin{tabular}{l|c|l}
    \hline
    $\ket{a_1}$ & & $\ket{a_1,d_1}$\\
    $\ket{a_2}$ & & \\
    $\ket{a_3}$ & & $\ket{a_2,d_2}$\\
    $\ket{r_1}$ & &\\
    $\ket{r_2}$ & \textbf{QRAM} & $\ket{a_3,d_3}$\\
    $\ket{r_3}$ & &\\
    $\ket{d_1}$ & & $\ket{d_1}$\\
    $\ket{d_2}$ & & $\ket{d_2}$\\
    $\ket{d_3}$ & & $\ket{d_3}$\\
    \hline
  \end{tabular}
  \normalsize
  \end{center}

}

%------------------------------------------------------------------------------
\frame[label=qram-grover]{

  \frametitle{Generalized Grover}

  What do we have to change in Grover's algorithm?

  \begin{enumerate}
  \item \textbf{Begin by reading the database mix state $\ket{D}$.}
  \item For each Grover iteration:
  \begin{enumerate}
  \item Query the oracle function $f$ as before.
  \item \textbf{Flip the marked states about the unmarked states as before.}
  \item Flip all states about $\ket{D}$.
  \end{enumerate}
  \item Measure the register.
  \end{enumerate}

  \begin{itemize}
  \item Grover's oracle $f$ is a black box, which we can ``fill in''.
  \item We can design an $f$ for many relational operations.
  \end{itemize}

}

%%%%%%%%%%%%%%%%%%%%%%%%%%%%%%%%%%%%%%%%%%%%%%%%%%%%%%%%%%%%%%%%%%%%%%%%%%%%%%%
\section{Algorithms}

%------------------------------------------------------------------------------
\frame[label=alg-ss]{

  \frametitle{Subset Search Example}

   \footnotesize
   \begin{center}
    \begin{tabular}{|l|l|}
      \hline
      \textbf{Item ID} & \textbf{Group ID}\\
      \hline
       Al Gore & Wikipedia\\ \hline
       Al Gore & Rotten Tomatoes\\ \hline
       Al Gore & whitehouse.gov\\ \hline
       Democrats & whitehouse.gov\\ \hline
       Democrats & Wikipedia\\ \hline
       Global warming & Nature\\ \hline
       Global warming & Rotten Tomatoes\\
      \hline
    \end{tabular}

   \vspace{\baselineskip}

    \textbf{Searched Subset}\\
    \begin{tabular}{|l|}
    \hline
      Al Gore \\ \hline
      Global warming\\ \hline
    \end{tabular}\\

   \vspace{\baselineskip}

    \textbf{Result}\\
    \begin{tabular}{|l|}
    \hline
      Rotten Tomatoes\\ \hline
    \end{tabular}
  \end{center}
  \normalsize

}

%------------------------------------------------------------------------------
\frame[label=alg-ss-2]{

  \frametitle{Subset Search Bounds}

   You are given:
   \begin{itemize}
     \item A table $G$ of $g$ group IDs.
     \item A table $I$ of $N$ items with foreign keys into $I$.
   \end{itemize}

   How to find all groups containing a particular subset of $k$ items?

   \textbf{Approach:} Store an $N$-bit vector with each of $g$ group IDs.
   Each bit is 1 if the corresponding item belongs to that group and
   0 otherwise. $f$ ANDs the item bits restricted to the
   subset.

   \vspace{0.5\baselineskip}

   \begin{itemize}
     \item Classical: $O(gk)$ time, $O(gN)$ space.
     \item Quantum: $O(g\sqrt{kt})$ time, $O(gN)$ space.
   \end{itemize}

   \vspace{0.5\baselineskip}

   \textbf{Applications:} web search, feature tagging databases.
}

%------------------------------------------------------------------------------
\frame[label=alg-nn]{

  \frametitle{Nearest Neighbor Example}

   \footnotesize
   \textbf{Database}
   \begin{tabular}{|l|l|l|l|l|l|l|}
   \hline
   \textbf{Name} & \textbf{Tall} & \textbf{Funny} & \textbf{Super} &
                   \textbf{Famous} & \textbf{Genius} & \textbf{Fictional} \\
   \hline
   Clark Kent  & Yes & No  & Yes & No  & No & Yes\\ \hline
   Jon Stewart & No  & Yes & No  & Yes & No & No\\ \hline
   Stephen Hawking & No & No & No & Yes & Yes & No\\ \hline
   \end{tabular}

   \vspace{0.5\baselineskip}

   \textbf{Query}
   \begin{tabular}{|l|l|l|l|l|l|}
   \hline
   \textbf{Tall} & \textbf{Funny} & \textbf{Super} &
                   \textbf{Famous} & \textbf{Genius} & \textbf{Fictional} \\
   \hline
   Don't care & Yes & Yes & Don't care & Don't care & No\\ \hline
   \end{tabular}

   \vspace{0.5\baselineskip}

   \textbf{Result}
   \begin{tabular}{|l|}
   \hline
   \textbf{Name}\\ \hline
   Jon Stewart \\ \hline
   \end{tabular}
   \normalsize

   \vspace{0.5\baselineskip}

   You are given:
   \begin{itemize}
     \item A table $N$ tuples, each with $k$ binary attributes.
   \end{itemize}

   \vspace{0.5\baselineskip}

   How do you find the ``neighbor'' with the fewest differences from
   a particular input tuple?
}

%------------------------------------------------------------------------------
\frame[label=alg-nn-2]{

  \frametitle{Nearest Neighbor Bounds}

   \textbf{Approach:} $f$ computes the Hamming distance restricted to
   fields we care about. Then use quantum search on the distances to
   find the minimum [DH96].

   \vspace{0.5\baselineskip}

   \begin{itemize}
     \item Classical: $d\log{N}/\min{(\epsilon^2,1)}$ update and query time,
$N^{O(1/\epsilon^2 + \log{(1+\epsilon)}/(1+\epsilon))}$ space [HIM03].
     \item Quantum: $O(d\sqrt{N}))$ query time, $O(1)$ update time, $O(1)$ space.
   \end{itemize}

   \vspace{0.5\baselineskip}

   \textbf{Applications:} machine learning, probabilistic databases.
}

%%%%%%%%%%%%%%%%%%%%%%%%%%%%%%%%%%%%%%%%%%%%%%%%%%%%%%%%%%%%%%%%%%%%%%%%%%%%%%%
\section{Future}

%------------------------------------------------------------------------------
\frame[label=future-skepticism]{

  \frametitle{Skepticism}

  \begin{itemize}
    \item Will cheap, fast QRAMs ever be feasible?
      \begin{itemize}
        \item Not without compelling applications.
      \end{itemize}
    \item Don't classical databases already scale for realistic datasets?
      \begin{itemize}
        \item Not for future quantum data
              (entangled crypto qubits, intermediate computation, quantum software).
        \item Most classical applications did not exist before the technology
              (bootstrapping problem).
      \end{itemize}
  \end{itemize}
}

%------------------------------------------------------------------------------
\frame[label=future-work]{

  \frametitle{Future Work}

  \begin{itemize}
    \item
    Do aggregation with quantum counting.
    \item 
    Use quantum data structures or complex amplitudes
    to improve upper bounds given.
    \item
    Find an efficient quantum join (naively in $O(N^{3/2})$, 
    must beat classical $O(N\log{N})$).
    \item
    Combine with the Quantum Fourier transform to do search on
    2D images, video, audio, etc.
  \end{itemize}
}

\end{document}



%%% Local Variables: 
%%% mode: latex
%%% TeX-master: "beamerexample2.article"
%%% End: 
