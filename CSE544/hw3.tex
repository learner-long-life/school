\documentclass[12pt]{article}

\usepackage{fullpage}
\usepackage{fancyhdr}
\usepackage{cse-scribe}
\pagestyle{fancy}

\rhead{Problem \thesection\\Page \thepage\\Spring 2006}
\lhead{Paul Pham [ppham@cs.washington.edu]\\CSE 544: Databases\\Homework 3}

%\renewcommand{\headrulewidth}{0.5pt}
\renewcommand{\footrulewidth}{0pt}

\addtolength{\headheight}{42pt} % make space for the rule
\addtolength{\headsep}{6pt} % make space for the rule

\renewcommand{\labelenumi}{(\alph{enumi})}
\renewcommand{\labelenumii}{\roman{enumii}.}
%\renewcommand{\thesection}{\small{Problem \arabic{section}.}}
%  \makeatletter
%   \renewcommand{\section}{\@startsection{section}{1}{0mm}
%   {\baselineskip}%
%   {\baselineskip}{\normalfont\normalsize}}%
%   \makeatother
%\renewcommand{\section}{\@startsection{section}{1}}
\def\qopnamewl@#1{\mathop{\fam\z@#1}\nlimits@}
\def\Exp#1{\mathop{\rm {E}}[#1]}
\def\argmax{{\rm arg} \mathop{\rm {max}}}
\def\argmin{{\rm arg} \mathop{\rm {min}}}
\def\dist{{\rm dist}\,}
\def\PR#1{\mathop{\rm {Pr}}\left[#1\right]}

\newenvironment{lscommand}%
    {\nopagebreak\par\small\addvspace{3.2ex plus 0.8ex minus 0.2ex}%
     \vskip -\parskip
     \noindent%
     \begin{tabular}{|l|}\hline\rule{0pt}{1em}\ignorespaces}%
    {\\\hline\end{tabular}\par\nopagebreak\addvspace{3.2ex plus 0.8ex
        minus 0.2ex}%
     \vskip -\parskip}

\begin{document}

%%%%%%%%%%%%%%%%%%%%%%%%%%%%%%%%%%%%%%%%%%%%%%%%%%%%%%%%%%%%%%%%%%%%%%%%%%%%%%%
% Problem 1
\section{Beers-bars-drinkers problem}

To ensure safety, we restrict the variables below to the active domain,
where

\begin{displaymath}
adom(d) = \exists{r} F(d,r) \land \exists{b} L(d,b) \land \forall{r} \exists{b} S(r,b)
\end{displaymath}

\begin{enumerate}
%------------------------------------------------------------------------------
\item % Part A
Find all drinkers that frequent only bars that serve only beer they
like. (Optimists)

\begin{displaymath}
q(d)\textrm{ } :- \textrm{ } adom(d) \land \lnot [\exists{r} [F(d,r) \land \exists{b}[S(r,b)\land \lnot L(d,b) ]]]
\end{displaymath}

%------------------------------------------------------------------------------
\item % Part B
Find all drinkers that frequent only bars that serve some beer they
like. (Realists)

\begin{displaymath}
q(d)\textrm{ } :- \textrm{ } adom(d) \land \lnot [\exists{r}.\exists{b}. F(d,r) \land L(d,b) \land \lnot S(r,b)]
\end{displaymath}

%------------------------------------------------------------------------------
\item % Part C
Find all drinkers that frequent some bar that serves only beers
they like. (Prudents)

\begin{displaymath}
q(d)\textrm{ } :- \textrm{ } adom(d) \land \exists{r} [F(d,r) \land \lnot [\exists{b}. S(r,b) \land \lnot L(d,b)]]
\end{displaymath}

%------------------------------------------------------------------------------
\item % Part D
Find all drinkers that frequent only bars that serve none of the
beers they like. (Flagellators)

This includes drinkers who frequent no bars, and also drinkers who like
all beers and frequent no bars except those that do not serve any beers.

\begin{displaymath}
q(d)\textrm{ } :- \textrm{ } adom(d) \land \lnot \exists{r} [F(d,r) \land \exists{b} [L(d,b) \land S(r,b)]]
\end{displaymath}

\end{enumerate}

%%%%%%%%%%%%%%%%%%%%%%%%%%%%%%%%%%%%%%%%%%%%%%%%%%%%%%%%%%%%%%%%%%%%%%%%%%%%%%%
% Problem 2
\section{Range-restricted formulae}

\begin{enumerate}
%------------------------------------------------------------------------------
\item % Part A
False. Some range-restricted query may take the complement of a relation,
for example, and this would return different answers conjoined to the
active domain depending on the full domain.

%------------------------------------------------------------------------------
\item % Part B
True. If all variables must appear in the active domain, which is finite,
a range-restricted query can only return a finite answer.
%------------------------------------------------------------------------------
\item % Part C
False. A safe formula could just guarantee membership in one particular
relation rather than membership in some relation (only one term of the
disjunction in $adom(u)$.

%------------------------------------------------------------------------------
\item % Part D
True. Safe queries are domain-independent and only depend on the extent of
the relations (the active domain). Therefore, restricting variable values to
the active domain should not affect membership in the query answer.
%------------------------------------------------------------------------------
\item % Part E
False. The set of range-restricted queries is recursively-enumerable.

%------------------------------------------------------------------------------
\item % Part F
False. This was given in lecture.

\end{enumerate}

\pagebreak

%%%%%%%%%%%%%%%%%%%%%%%%%%%%%%%%%%%%%%%%%%%%%%%%%%%%%%%%%%%%%%%%%%%%%%%%%%%%%%%
% Problem 3
\section{Finite relations}

\begin{enumerate}
%------------------------------------------------------------------------------
\item % Part A
\begin{displaymath}
adom(x) = \exists(y) [R(x,y) \lor R(y,x) \lor U(x,y) \lor U(y,x)] \lor S(x)
\end{displaymath}
%------------------------------------------------------------------------------
\item % Part B
\begin{enumerate}
\item % i
Finite. Not safe.

\begin{displaymath}
\{x \mid adom(x) \land S(x) \land \forall{y} (adom(y) \Rightarrow \lnot R(x,y))\}
\end{displaymath}

\item % ii
Finite. Not safe.

\begin{displaymath}
\{x \mid adom(x) \land S(x) \land \forall{y} (adom(y) \Rightarrow (R(x,y) \Rightarrow \exists{z}(adom(z) \land(S(z) \lor U(y,z)))))\}
\end{displaymath}

\item % iii
Not finite. Not safe. No equivalent range-restricted formula because not finite.

\item % iv
Finite. Safe.

\begin{displaymath}
\{x \mid adom(x) \land S(x) \land \forall{y} (adom(y) \Rightarrow (S(y) \land R(x,y)))\}
\end{displaymath}

\item % v
Finite. Not safe.

\begin{displaymath}
\{x \mid adom(x) \land S(x) \land \forall{y}(adom(y) \Rightarrow (U(x,y) \lor \forall{z}.(adom(z) \Rightarrow \lnot R(y,z))))\}
\end{displaymath}

\item % vi
Finite. Safe.

\begin{displaymath}
\{(x,y) \mid adom(x) \land adom(y) \land \exists{z}.(adom(z) \land (R(x,z) \lor R(z,y)))\}
\end{displaymath}

\end{enumerate}
\end{enumerate}

%%%%%%%%%%%%%%%%%%%%%%%%%%%%%%%%%%%%%%%%%%%%%%%%%%%%%%%%%%%%%%%%%%%%%%%%%%%%%%%
% Problem 4
\section{Binary tree}

\begin{enumerate}
%------------------------------------------------------------------------------
\item % Part A
%
\begin{eqnarray*}
SameLevel(y,z) & :- & T(x,y,z)\\
SameLevel(y,z) & :- & T(u,y,w),T(v,z,t),SameLevel(u,v)\\
Answer(y,z) & :- & SameLevel(y,z)
\end{eqnarray*}
%
%------------------------------------------------------------------------------
\item % Part B
%
\begin{eqnarray*}
AliceWins(x) & :- & L(x),A(x)\\
AliceWins(x) & :- & T(x,y,z),AliceWins(y)\\
AliceWins(x) & :- & T(x,y,z),AliceWins(z)\\
Answer(x) & :- & AliceWins(x)
\end{eqnarray*}
%
\end{enumerate}

\pagebreak

%%%%%%%%%%%%%%%%%%%%%%%%%%%%%%%%%%%%%%%%%%%%%%%%%%%%%%%%%%%%%%%%%%%%%%%%%%%%%%%
% Problem 5
\section{Query containment}

\begin{enumerate}
%------------------------------------------------------------------------------
\item % Part A
\begin{enumerate}
\item % i
Yes, $q \subseteq q'$ with the following variable mapping function
$f: var(q') \rightarrow var(q)$:
$x \rightarrow x, y \rightarrow y, z \rightarrow z,
u \rightarrow x, v \rightarrow y$.

$body(q')$ maps to $body(q)$ as follows:
$R(x,y) \rightarrow R(x,y), R(y,z) \rightarrow R(y,z),
R(z,u) \rightarrow R(z,x), R(u,v) \rightarrow R(x,y),
R(v,z) \rightarrow R(y,z)$
%
\item % ii
No, $q \nsubseteq q'$ with the following counterexample:
%
\begin{eqnarray*}
R & = & \{(1,2,2), (2,3,4) (4,4,5)\}\\
D & = & \{1,2,3,4,5\}\\
I & = & (D, R)\\
\end{eqnarray*}
%
$q(I) = \{(1,5)\}$ but $q'(I) = \emptyset$.
%
\item % iii
Yes, $q \subseteq q'$. Note that both $q$ and $q'$ are boolean queries.
If $q(R)$ is true, then for some tuples $R(u,u,x,y),R(x,y,w,v)$ either
$(x=y)_q$ or $(x\ne y)_q$.

If $(x = y)_q$, then we can map $var(q')$ as follows: $u \rightarrow x,
x \rightarrow w, y \rightarrow v$.

If $(x \ne y)_q$, then we can map $var(q')$ as follows: $u \rightarrow u,
x \rightarrow x, y \rightarrow y$.

Therefore, if $q(R)$ is true, then $q'(R)$ must also be true, and therefore
the first is a subset of the latter.

\item % iv
%
No, $q \nsubseteq q'$ with the following counterexample:
%
\begin{eqnarray*}
R & = & \{(1,1)\}\\
D & = & \{1\}\\
I & = & (D, R)\\
\end{eqnarray*}
%
$q(I) = \{(1)\}$ but $q'(I) = \emptyset$.
%
\end{enumerate}
%------------------------------------------------------------------------------
\item % Part B
%
\begin{displaymath}
q(x) \textrm{ } := \textrm{ } R(x,y),R(y,z),R(v,w)
\end{displaymath}
%
Then we can see that $q_1 \subseteq q$ and $q \subseteq q_2$.
In fact, $q \equiv q_2$.

\pagebreak
%------------------------------------------------------------------------------
\item % Part C

\begin{displaymath}
q(x) \textrm{ } := \textrm{ } R(x,a),R(a,b),R(b,u)
\end{displaymath}

Then we have the following variable mapping function
$f: var(q_1) \rightarrow var(q)$:
$x \rightarrow x, y \rightarrow  a, z \rightarrow b$.

The corresponding body of the queries are mapped as follows:
$R(x,y) \rightarrow R(x,a), R(y,z) \rightarrow R(a,b),
R(a,z) \rightarrow R(a,b)$.

$t_q = x$ and $x \in q_1(D_q)$, so we see that $q \subseteq q_1$.

We also have the following variable mapping function
$f: var(q_2) \rightarrow var(q)$:
$x \rightarrow x, y \rightarrow a, z \rightarrow b, u \rightarrow b$.

The corresponding body of the queries are mapped as follows:
$R(x,y) \rightarrow R(x,a), R(y,z) \rightarrow R(a,b),
R(z,u) \rightarrow R(b,u), R(y,b) \rightarrow R(a,b)$.

$t_q = x$ and $x \in q_2(D_q)$, so we see that $q \subseteq q_2$.

Let

\begin{displaymath}
q'(x) \textrm{ } := \textrm{ } R(x,y),R(y,z)
\end{displaymath}

Then the variable mapping function $f: var(q') \rightarrow var(q_1)$ is
identity, and likewise for the predicates $R(x,y)$ and $R(y,z)$.

$t_{q'} = x$ and $x \in q'(D_{q_1})$, so we see that $q_1 \subseteq q'$.

We also have the variable mapping function $f: var(q') \rightarrow var(q_2)$
as identity, and likewise for predicates $R(x,y)$ and $R(y,z)$.

$t_{q'} = x$ and $x \in q'(D_{q_2})$, so we see that $q_2 \subseteq q'$.

\end{enumerate}

%%%%%%%%%%%%%%%%%%%%%%%%%%%%%%%%%%%%%%%%%%%%%%%%%%%%%%%%%%%%%%%%%%%%%%%%%%%%%%%
% Problem 6
\section{True-false}

\begin{enumerate}
%------------------------------------------------------------------------------
\item % Part A
True. P is exactly equivalent to FO with least fixed-point operators,
so the descriptive complexity of FO is a subset of P.
%------------------------------------------------------------------------------
\item % Part B
False, FO queries containing negation/complement are not monotone.
%------------------------------------------------------------------------------
\item % Part C
False. The query complexity of CQs is exponential in the size of the
query.
%------------------------------------------------------------------------------
\item % Part D
False. Standard Datalog contains recursion but no negation,
so any FO queries containing negation cannot be expressed in Datalog.
%------------------------------------------------------------------------------
\item % Part E
True. If a query can be expressed in FO, it contains no recursion. If a
query can be expressed in Datalog, it contains no negation. UCQ is exactly
equivalent to Datalog with no recursion or negation.

\end{enumerate}

\end{document}
