\documentclass[10pt]{beamer}
\mode<beamer>{%
\usetheme[hideothersubsections,right,width=22mm]{Goettingen}
}
\title{Introduction to PPAD [Papadimitriou 1992]}
\author[P. Pham]{Paul Pham}
\institute{UW CSE}
%\titlegraphic{\includegraphics[width=20mm]{USTL}}
\date{2006}

\usepackage{eepic}
\usepackage{color}
\usepackage{rotating}
\usepackage{latexsym} % For \Diamond

\begin{document}
\begin{frame}<handout:0>
\titlepage
\end{frame}

%==============================================================================
\section{Introduction}

%------------------------------------------------------------------------------
\begin{frame}
\frametitle{Overview}

Motivation:
Finding a mixed-strategy Nash equilibrium in general-sum games is in PPAD.

\begin{itemize}
\item Define function complexity classes FP, FNP, and TFNP.
\item Define interesting, closed subclasses of TFNP (PLS, PPA, PPAD).
\item Provide an example of PPA: $SMITH$.
\item Provide an example of PPAD: $2D-SPERNER$.
\item At the end: ready to show $NASH \le_P BROUWER \le_P 2D-SPERNER$
\end{itemize}

\end{frame}

%==============================================================================
\section{FNP, TFNP, PLS}

%------------------------------------------------------------------------------
\begin{frame}
\frametitle{Complexity Classes for Multi-valued Functions}

Classes of \textbf{function problems} which correspond to
\textbf{decision problems}:\\

FP = function polytime, induced by P

\begin{displaymath}
P(x,y) \in \textrm{FP} \textrm{ if for some det. PTM, } P(x,M(x)) holds
\end{displaymath}

FNP = function nondeterministic polytime, induced by NP

\begin{displaymath}
\displaystyle
P(x,y) \in \textrm{FNP} \textrm{ if for some nondet. PTM, } M(x,y) = P(x,y)
\end{displaymath}

Natural correspondence between NP-complete problems and FNP-complete problems.

%\begin{displaymath}
%\textrm{FP \subseteq FNP} \Leftrightarrow \textrm{P \subseteq NP}
%\end{displaymath}

\begin{displaymath}
\textrm{FP = FNP} \Leftrightarrow \textrm{P = NP}
\end{displaymath}

\end{frame}

%------------------------------------------------------------------------------
\begin{frame}
\frametitle{Total Functions}

\textbf{Total functions} always have a solution.

TFNP = total function nondeterministic polytime

\begin{displaymath}
\textrm{FP} \subseteq \textrm{\large{\textbf{TFNP}}} \subseteq \textrm{FNP}
\end{displaymath}

Examples:
\begin{itemize}
\item
\end{itemize}

\end{frame}

%------------------------------------------------------------------------------
\begin{frame}
\frametitle{Semantic vs. Syntactic Classes}

\textbf{Syntactic classes}

\begin{itemize}
\item Recursively enumerable
\item Closed under reduction (have complete problems)
\item Examples: P, NP (with polynomial clock)
\end{itemize}

\textbf{Semantic classes}
\begin{itemize}
\item Defined by promises, or behavior on inputs.
\item Not enumerable.
\item In general, no complete problems
\item Examples: BPP, RP, TFNP
\end{itemize}

We want to define syntactic subclasses of TFNP which still contain
important problems.

\end{frame}

%------------------------------------------------------------------------------
\begin{frame}
\frametitle{Class Hierarchy for PPAD}

\begin{figure}[hbt]
  \input{complexity}
  \centerline{\box\graph}
\end{figure}

\end{frame}

%------------------------------------------------------------------------------
\begin{frame}
\frametitle{Polynomial Local Search (PLS) Graph}

\begin{quote}
Every finite dag has a sink.
\end{quote}

\begin{figure}[hbt]
  \input{dummy}
  \centerline{\box\graph}
\end{figure}

\end{frame}

%------------------------------------------------------------------------------
\begin{frame}
\frametitle{PLS: Polynomial Local Search}

\begin{itemize}
\item \textbf{Input:} $x =$ a search graph
\item \textbf{Node space:} $S(x) = \Sigma^{p(|x|)}$
\item \textbf{Polytime algorithms:} $N(x,s)$ neighbor function, $c(x,s)$ cost function, $s \in S(x)$
\item Polynomially many neighbors: $|N(x,s)| \le |x|^k$
\item \textbf{Goal:} find a node with lower cost than its neighbors (local minima)
\end{itemize}

Implicitly encodes exponential-size dag in polynomial-size machine.

\begin{displaymath}
\textrm{FP} \subseteq \textrm{\large{\textbf{PLS}}} \subseteq \textrm{TFNP}
\end{displaymath}

\end{frame}

%==============================================================================
\section{PPA and SMITH}

%------------------------------------------------------------------------------
\begin{frame}
\frametitle{Polynomial Parity Argument (PPA) Graph}

\begin{quote}
Every finite graph has an even number of odd-degree nodes.
\end{quote}

\begin{figure}[hbt]
  \input{dummy}
  \centerline{\box\graph}
\end{figure}

\end{frame}

%------------------------------------------------------------------------------
\begin{frame}
\frametitle{PPA: Polynomial Parity Argument for 2-degree}

\begin{itemize}
\item On input $x$:
\item \textbf{Node space:} $S(x) = \Sigma^{p(|x|)}$
\item \textbf{Polytime algorithms:} $N(x,s)$ neighbor function, $s \in S(x)$
\item Cost function from PLS is uniform.
\item $|N(x,s)| \le 2$
\item \textbf{Goal:} Find a leaf other than the standard leaf $(0\cdots 0)$.
\end{itemize}

\begin{displaymath}
\textrm{\large{\textbf{PPA}}} \subseteq \textrm{PLS} \subseteq \textrm{TFNP} \subseteq
\textrm{FNP}
\end{displaymath}

\end{frame}
%------------------------------------------------------------------------------
\begin{frame}
\frametitle{PPA': Polynomial Parity Argument for $n$-degree}

$n$-degree graphs equivalent to $2$-degree graphs by construction:

\begin{itemize}
\item At each node, divide the $k$ degrees into $\lfloor \frac{k}{2} \rfloor$
      pairs, $k \le n$.
\item Create a new node for each pair with degree 2.
\item If $k$ is odd, then one new node is a leaf.
\end{itemize}

This preserves the number of odd-degree nodes in a graph.
Without loss of generality, consider only 2-degree graphs.

\begin{figure}[hbt]
  \input{dummy}
  \centerline{\box\graph}
\end{figure}

\begin{displaymath}
PPA' = PPA
\end{displaymath}

\end{frame}
%------------------------------------------------------------------------------
\begin{frame}
\frametitle{SMITH: Finding a second Ham-cycle}

\begin{itemize}
\item \textbf{Input:} $x =$ odd-degree graph promised to have one Ham-cycle.
\item \textbf{Node space:} $S(x) = $ rotations of Ham-paths with one fixed endpoint.
\item \textbf{Neighbors:} $N(x,s) = s'$ if $s$ and $s'$ differ by one edge.
\item $|N(x,s)| \le 2$, edges are undirected (symmetric)
\item \textbf{Goal:} Find a second Ham-cycle.
\end{itemize}

\begin{figure}[hbt]
  \input{dummy}
  \centerline{\box\graph}
\end{figure}

\end{frame}

%------------------------------------------------------------------------------
\begin{frame}
\frametitle{SMITH: Finding a second Hamilton cycle}

On a Ham-cycle, pick arbitrary edge $(u,v)$.

\begin{itemize}
\item Delete $(u,v)$ to create starting Ham-path.
\item Fix one endpoint ($u$).
\item While $v$ is not an endpoint, rotate other edges to create new Ham-path.
\end{itemize}

\begin{figure}[hbt]
  \input{dummy}
  \centerline{\box\graph}
\end{figure}

\end{frame}

%==============================================================================
\section{PPAD and 2D-SPERNER}

%------------------------------------------------------------------------------
\begin{frame}
\frametitle{PPAD: Polynomial Parity Argument Directed}

\begin{itemize}
\item Same inputs, node space, and polytime algorithms as PPA except
      neighbors are \textbf{directed}.
\item \textbf{Goal:} find a source or sink other than standard source.
\end{itemize}

Characterized by lemma:

\begin{quote}
Every finite dag has an even number of source-sinks.
\end{quote}

\begin{displaymath}
\textrm{\large{\textbf{PPAD}}} \subseteq \textrm{PPA} \subseteq \textrm{PLS} \subseteq
\textrm{TFNP} \subseteq \textrm{FNP}
\end{displaymath}

\end{frame}

%------------------------------------------------------------------------------
\begin{frame}
\frametitle{2D-SPERNER: Finding a trichromatic triangle}

\
\textbf{Sperner's Lemma:} Every admissible coloring of a triangulation has
a trichromatic triangle.

\textbf{Admissible coloring:} vertices of big triangle are colored 0, 1, and 2.
Edge $ij$ of big triangle does not contain color $3-i-j$.

\textbf{Comp. problem:} Given an admissible coloring, find a trichromatic
triangle.

\begin{figure}[hbt]
  \input{dummy}
  \centerline{\box\graph}
\end{figure}

\end{frame}
%------------------------------------------------------------------------------
\begin{frame}
\frametitle{kD-SPERNER: Generalizing to higher-dimensions}

Any $k-1$ points form a surface in $k$-dimensions called a \textbf{simplex}.

\begin{itemize}
\item Triangle $\Rightarrow$ simplex
\item Triangulation (subdivision into triangles) $\Rightarrow$ Simplicization
\item Trichromatic triangle $\Rightarrow$ panchromatic simplex
\end{itemize}

Examples:

\begin{itemize}
\item 1-simplex = point
\item 2-simplex = line
\item 3-simplex = triangle
\item 4-simplex = tetrahedron
\end{itemize}

Sperner's theorem can be proven for $k$-dimensions by induction on $k$.

\end{frame}
%------------------------------------------------------------------------------
\begin{frame}
\frametitle{2D-SPERNER: Rectangular form}

No easy way to subdivide a tetrahedron (4-simplex) $\Rightarrow$\\
Hard to generalize 2D to 3D (and beyond).
\vspace{\baselineskip}

\textbf{Solution:} Simulate a $d$-simplex with a $(d-1)$-dimensional hypercube
\vspace{\baselineskip}

\textbf{2D Case:}
\begin{itemize}
\item Three vertices of big square are colored 0, 1, and 2.
\item Remaining vertex is not colored 0.
\end{itemize}
\vspace{\baselineskip}

\begin{figure}[hbt]
  \input{dummy}
  \centerline{\box\graph}
\end{figure}

%\begin{displaymath}
%\xymatrix{
%2 \ar@{-}[r] \ar@{-}[d] & 1 \ar@{-}[r] \ar@{-}[d] & 2 \ar@{-}[d]\\
%0 \ar@{-}[r] \ar@{-}[d] & 0 \ar@{-}[r] \ar@{-}[d] & 2 \ar@{-}[d]\\
%0 \ar@{-}[r] & 1 \ar@{-}[r] & 1 }
%\end{displaymath}

\end{frame}
%------------------------------------------------------------------------------
\begin{frame}
\frametitle{2D-SPERNER: Square simulating a triangle}

\begin{figure}[hbt]
  \input{dummy}
  \centerline{\box\graph}
\end{figure}

%\begin{displaymath}
%\xymatrix{
%2 \ar@{-}[ddd] \ar@{-}[dddr] \ar@{-}[dddrr]\\
%\\
%\\
%2 \ar@{-}[r] \ar@{-}[d] & 1 \ar@{-}[r] \ar@{-}[d] & 2 \ar@{-}[d]\\
%0 \ar@{-}[r] \ar@{-}[d] & 0 \ar@{-}[r] \ar@{-}[d] & 2 \ar@{-}[d]\\
%0 \ar@{-}[r] \ar@{-}@/^1pc/[uu] \ar@{-}@/^/[u] \ar@{-}@/_/[r] \ar@{-}@/_1pc/[rr] &
%1 \ar@{-}[r] & 1 & & & 1 \ar@{-}[lll] \ar@{-}[ulll] \ar@{-}[uulll]}
%\end{displaymath}

\end{frame}

%==============================================================================
\section{Conclusion}

\begin{frame}
\frametitle{Conclusion}

\begin{itemize}
\item TFNP (semantic class) not closed under reduction.
\item Syntactic subclasses with important problems: PLS, PPA, PPAD.
\item $\textrm{PPA}' = \textrm{PPA}$
\item $SMITH \in \textrm{PPA}$
\item $2D-SPERNER \in \textrm{PPAD}$
\item Ready to show $NASH \le_P BROUWER \le_P 2D-SPERNER$
\end{itemize}
\end{frame}

\end{document}
