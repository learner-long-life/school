\documentclass[12pt]{article}

\usepackage{fullpage}
\usepackage{fancyhdr}
\usepackage{cse-scribe}
\pagestyle{fancy}

\rhead{Problem \thesection\\Page \thepage\\Winter 2006}
\lhead{Paul Pham\\ppham@cs.washington.edu\\CSE 521 Problem Set 3}

%\renewcommand{\headrulewidth}{0.5pt}
\renewcommand{\footrulewidth}{0pt}

\addtolength{\headheight}{42pt} % make space for the rule
\addtolength{\headsep}{6pt} % make space for the rule

\renewcommand{\labelenumi}{\textbf{\alph{enumi})}}
\renewcommand{\labelenumii}{\textbf{\arabic{enumii}}}
%\renewcommand{\thesection}{\small{Problem \arabic{section}.}}
%  \makeatletter
%   \renewcommand{\section}{\@startsection{section}{1}{0mm}
%   {\baselineskip}%
%   {\baselineskip}{\normalfont\normalsize}}%
%   \makeatother
%\renewcommand{\section}{\@startsection{section}{1}}
\def\qopnamewl@#1{\mathop{\fam\z@#1}\nlimits@}
\def\Exp{\mathop{\rm {E}}}
\def\dist{{\rm dist}\,}

\begin{document}

%%%%%%%%%%%%%%%%%%%%%%%%%%%%%%%%%%%%%%%%%%%%%%%%%%%%%%%%%%%%%%%%%%%%%%%%%%%%%%%
% Problem 1
\section{Fair carpooling}

\begin{enumerate}
%------------------------------------------------------------------------------
\item % Part A
%
Construct a bipartite graph $G = (V = X \cup Y, E)$ as follows:
\begin{itemize}
\item A node $x_j \in X$ for every driver $p_j$. ($|X|=k$).
\item A node $y_i \in Y$ for every carpool day $S_i$. ($|Y|=d$).
\item An edge $(x_j,y_i) \in E$ with capacity 1 if $p_j \in S_i$.
\item An edge from the source $s$ to $x_j$ with capacity $\lceil \Delta_j \rceil$.
\item An edge from each $y_i$ to the sink $t$ with capacity 1.
\end{itemize}
%
We know that the sum of all the driving obligations is $d$ from the following.
%
\begin{displaymath}
\sum_{j=1}^{k} \Delta_j = \sum_{j=1}^{k} \sum_{i:p_j \in S_i}
\frac{1}{|S_i|} = \sum_{i=1}^{d} \sum_{j:p_j \in S_i} \frac{1}{|S_i|}
= \sum_{i=1}^{d} 1 = d
\end{displaymath}
%
We also know:
$\sum_{j=1}^{k} \lceil \Delta_j \rceil \ge \sum_{j=1}^{k} \Delta_j = d$
and that $|E| \ge d$ since every carpool day must have at least one
person. So the smallest $s-t$ cut in the graph is on the edges between $Y$
and $t$, and the max flow equals the capacity of this min cut ($d$).
With integer flows, we can read off the drivers for each
day as all the edges $(x_j,y_i)$ with flow 1, where person $p_j$ is the
driver for the $i$th day. There are $d$ such edges, guaranteeing a driver
for every day. To conserve flow out of each $y_i$, each day has at most
one driver.
The capacities on edges from $s$ guarantee that obligations
are not exceeded. Therefore, a fair driving schedule exists for every
set of drivers and carpool days.

%------------------------------------------------------------------------------
\item % Part B
Find the max flow / min cut on the graph formed in Part (a) using
Ford-Fulkerson. The sum of capacities out of $s$ is at most $2d$, and in
the worst case every driver rides every day of the carpool ($|E|=kd$).
Ford-Fulkerson, using a DFS/BFS method for augmenting paths, runs in
time bounded by the product of source capacity and
$O(|V|+|E|) = O((k+d)+kd) = O(kd)$
So this runs in time $O(kd^2)$.

%------------------------------------------------------------------------------
\item % Part C
We can exploit the structure of our graph to choose a smarter way of
augmenting paths than simple DFS/BFS as in the previous part.

Namely, once a day $y_i \in Y$ has been assigned a driver
$x_j \in X$ there is no way for another $s-t$ path to go through
$y_i \leadsto t$.
In each of the $d$ steps of the Ford-Fulkerson algorithm, we increase
the flow by at exactly one by assigning a driver $x_j$ to $y_i$.
Then we should also remove all residual graph edges corresponding to other
possible drivers for that day: $x_{j'} \in S_i$, $x_{j'} \ne x_j$.
In the worst case, all drivers ride on
all days, so this edge removal takes $O(k)$ time.
For any driver $x_{j'}$ in
the residual graph, any remaining residual edge leads to a day node
without a driver, which is guaranteed to be a new $s-t$ path.

Therefore, in each of $d$ BFS augmentations,
it takes $O(k)$ to find a path to a driver $x_j$ whose
obligation is not exceeded (capacity on $s \leadsto x_j$ path)
and who has
an edge to some $y_i \leadsto t$ path and $O(k)$ to remove all other
driver edges.
The final running time is then $O(k^2d)$.

\end{enumerate}

\pagebreak

%%%%%%%%%%%%%%%%%%%%%%%%%%%%%%%%%%%%%%%%%%%%%%%%%%%%%%%%%%%%%%%%%%%%%%%%%%%%%%%
% Problem 2
\section{Network failures}

Find the $k$ disjoint paths using the algorithm in Section 7.6 of the
book. This takes $O(mn)$ time. For each of the $k$ disjoint paths,
use a binary search approach of pinging nodes along the path,
reducing the search space in half at each step. In the worst case,
all paths have length $O(n)$, and the binary search takes $O(\log{n})$ pings
for each path.
Therefore this algorithm uses $O(k\log{n})$ pings.

\pagebreak

%%%%%%%%%%%%%%%%%%%%%%%%%%%%%%%%%%%%%%%%%%%%%%%%%%%%%%%%%%%%%%%%%%%%%%%%%%%%%%%
% Problem 3
\section{Graph density}
\begin{enumerate}
%------------------------------------------------------------------------------
\item % Part A
Given an undirected graph $G=(V,E)$, construct the following directed graph
$G'=(V',E')$:

\begin{itemize}
\item
Create a source $s$ and connect it with a directed edge \textit{to}
every $v \in V$ with capacity $\alpha$.
\item
Create a sink $t$ and connect to it \textit{from} every $v \in V$
with capacity $deg(v)/2$.
\item
For every $(u,v) \in E$, create two directed edges $(u,v),(v,u) \in E'$
with capacity $\frac{1}{2}$.
\end{itemize}
%
Run the Ford-Fulkerson algorithm to find a min $s-t$ cut on $G'$, which will
divide $V'$ into $S'$ (containing $t$) and $V'/S'$ (containing $s$) to
minimize the sum of the following capacities:
%
\begin{enumerate}
\item Edge capacities from $s$ to every node in $S'$ to get $\alpha|S'|$.
\item Edge capacities from every node in $V'/S'$ to $t$ to get the number
of edges internal to $V'/S'$, $E(V'/S')$.
\item Edge capacities which cross \textit{from} $V'/S'$ \textit{to} $S'$ to
get the number of edges that are not internal to either $S'$ or $V'/S'$.
\end{enumerate}
%
We then use the following facts: minimizing (1) above means maximizing its
negative, and minimizing the sum of (2) and (3) above means maximizing
its complement in $|E'|$. Therefore we have:
\begin{displaymath}
\displaystyle \min_{S' \subseteq V'}\left[\alpha|S'| + \sum_{v \in V'/S'}\frac{1}{2}deg(v) +
\sum_{\begin{subarray}{l}
(u,v) \in E'\\u \in V'/S'\\v \in S' \end{subarray}}\frac{1}{2}\right] =
%\max_{S' \subseteq V'}\left[-\alpha|S'| + \sum_{v \in S'}\frac{1}{2}deg{v}\right] =
\max_{S' \subseteq V'}\left[-\alpha|S'| + E(S')\right]
\end{displaymath}
Therefore the min cut returns a subset $S' \subseteq V'$ containing $t$
whose corresponding $S \subseteq V$ maximizes the desired quantity.
%------------------------------------------------------------------------------
\item % Part B
Maximizing $|E(S)|-\alpha|S|$ is equivalent to testing if a graph contains
a maximum density $\alpha$. We know the density must be 
rational because it is defined as the ratio of a number of edges and
a number of vertices, both integers.

The maximum value of $|E(S)|$ is $\binom{|S|}{2} = O(|S|^2)$,
in the case of a completely
connected component. Therefore, the density of any component is
$O(|S|^2/|S|) = O(|S|)$, and the maximum vertex subset is all of
$V$.
Using a binary search strategy, we can call our algorithm from part (a),
which uses Ford-Fulkerson to find the max flow,
with a search space of $O(|V|)$ rational values,
since the least common multiple of all density values is at worst $|V| = n$.
This takes $O(\log{n})$ max flow executions as required,
with the initial range of the
search from $0$ to $|E|/|V|$.
After each ``maximization test'', if
the maximum value is negative, then we know that no subset $S$
has density $\alpha$ and we pivot to a lower value, otherwise we pivot
to a higher value (both rounded to the nearest multiple of $1/|V|$).
When the binary search terminates, return the highest density found.

\end{enumerate}
\pagebreak

%%%%%%%%%%%%%%%%%%%%%%%%%%%%%%%%%%%%%%%%%%%%%%%%%%%%%%%%%%%%%%%%%%%%%%%%%%%%%%%
% Problem 4
\section{BFS augmentation}

\begin{enumerate}
%------------------------------------------------------------------------------
\item % Part A
In updating the residual graph, we either add a back edge or decrease the
capacities of existing edges (and possibly remove them).
Decreasing capacities or removing edges
does not change the shortest hop distance $(s,v)$ and may increase it if
the removed edge was on an existing shortest path.

Therefore, we argue that adding backedges cannot decrease shortest paths.
Suppose that after taking a forward edge $(u,w)$ in
$G_f$, we add a back edge $(w,u)$ that
decreases some $(s,y)$ path in $G_{f'}$. The new
shortest path is now
$s \leadsto w \leadsto u \leadsto y$. However, in that case
a shorter path already existed before the back edge:
$s \leadsto u \leadsto y$. This either still exists in $G_{f'}$ or has been
disconnected since $G_f$, so any path that BFS augmentation finds later
can be no shorter.
%------------------------------------------------------------------------------
\item % Part B
In the previous part, we showed that the shortest path $(s,u)$ never
decreases while updating $G_f$. When $(u,v)$ is first a bottleneck, it
is on the shortest path $s \leadsto u \leadsto v \leadsto t$.
$d_f(s,v) \ge d_f(s,u) + 1$, otherwise we have a shorter path
$s \leadsto v \leadsto t$ that bypasses $u$. After being updated, the back edge
$(v,u)$ is added and $(u,v)$ removed (in $G_{f'}$).

If $(u,v)$ is a bottleneck for a second time in the future,
then previously $(v,u)$
must have been updated on a shortest path $s \leadsto v \leadsto u \leadsto t$.
Therefore,
$d_{f'}(s,u) \ge d_{f'}(s,v) + 1$, otherwise we have a shorter path
$s \leadsto u \leadsto t$ that bypasses $v$.

Therefore, $d_{f'}(s,u) \ge d_{f'}(s,v) + 1 \ge d_f(s,u) + 2$.

%------------------------------------------------------------------------------
\item % Part C
In each iteration, we are finding a bottleneck edge in the shortest $s-t$ path
and replacing it with a backedge of equal capacity; that is,
we remove at least one forward edge on each BFS traversal.
There are $|E|$ such edges that can be bottlenecks.
The maximum number of times that an edge $(u,v)$
can be a bottleneck is $|V|/2$ by the previous parts, since in the worst case
$d_f(s,u)$ starts as 1 and grows to $|V|$ in increments of $2$ every time
the edge becomes a bottleneck, never decreasing.
Therefore, the number of augmentations needed
is $O(|E||V|)$.

\end{enumerate}

\end{document}
